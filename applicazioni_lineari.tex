\section{Applicazioni lineari}
\subsection{Definizione: applicazione lineare (trasformazione lineare o homomorfism of vector space)}
Un'applicazione lineare è un'applicazione $T:V\rightarrow W$ di spazi vettoriali su $K$ taòe che $\forall v,v_1,v_2\in V$, $\forall\lambda\in K$ si ha:
\begin{enumerate}
	\item $T(v_1+v_2)=T(v_1)+T(v_2)$ ($T$ è additiva)
	\item $T(\lambda\cdot v)=\lambda\cdot T(v)$ ($T$ è omogenea)
\end{enumerate}
(additiva+omogenea = è lineare).
\\In altre parole $T$ conserva le operazioni di spazio vettoriale.

\subsection{Proposizione}
Punti $1+2\Leftrightarrow 3:\; \forall\lambda\mu\in K,\forall v_1,v_2\in V$
\[T(\lambda v_1+\mu v_2)=\lambda T(v_1)+\mu T(v_2)\]
$T$ manda combinazioni lineari in combinazioni lineari (con gli stessi scalari) dei trasformati.

\subsection{Definizione}
Fissata $A\in \mathcal{M}_{mn}(\mathbb{R}):$ 
$A=\begin{pmatrix}a_{11}&...&a_{1_n}\\&...&\\a_{n1}&...&a_{mn}\end{pmatrix}$, sia $L_A:\mathbb{R}^n\rightarrow\mathbb{R}^m$ definita da:
\[X\begin{pmatrix}x_1\\...\\x_n\end{pmatrix}\mapsto L_A(x):=AX=\begin{pmatrix} a_{11}x_1+a_{12}x_2+...+a_{1n}x_n\\...\\a_{m_1}x_1+a_{m_2}x_2+...+a_{mn}x_n\end{pmatrix}\in\mathbb{R}^n\]
La trasformazione $L_A$ è la moltiplicazione (a destra) per $A$:
\[L_A:X\mapsto AX\]
ad ogni matrice $A$ corrisponde una trasformazione lineare $L_A$.

\textit{Per esercizio: dimostrare che $L_A:\mathbb{R}^n\rightarrow\mathbb{R}^m$ è lineare.
\begin{itemize}
	\item $L_A(X+Y)=L_A(X)+L_A(Y)$? (dimostrazione su appunti)
	\item $L_A(\lambda X)=\lambda L_A(X)?$
\end{itemize}}

\subsection{Definizione}
Data una trasformazione lineare $T:V\rightarrow W$, restano determinati due sottoinsiemi:
\begin{itemize}
	\item Il nucleo di $T$: $KerT$
	\[KerT=\{v\in V: T(v)=0_W\}\subseteq V\]
	\item L'immagine di $T$: $ImT$
	\[ImT=\{T(v):v\in V\}\subseteq W\]
\end{itemize}

\subsection{Proposizione (risolvere per esercizio)}
\begin{enumerate}
	\item $KerT\leq V$
	\item $ImT\leq W$
	\item $T$ suriettiva $\Leftrightarrow ImT=W$
	\item $T$ iniettiva $\Leftrightarrow KerT=\{0\}$
\end{enumerate}

\subsection{Osservazione (!)}
Calcolare il nucleo di un'applicazione lineare corrisponde a risolvere un sistema omogeneo.

\subsection{Proposizione}
Un'applicazione lineare $T:V\rightarrow W$ è definita univocamente (ovvero è determinata) quando si conoscono le immagini di $T$ sui vettori $\{v_1,...,v_n\}$ di una base di $V$, se $dimV=n$: ovvero basta conoscere
\(T(v_1),...,T(v_n)\in W\) per conoscere tutta la trasformazione lineare $T$.
\\
\\\textbf{Dimostrazione}
Se conosco $T$ su $v_1,...,v_n$ allora $v$ si scrive come $v=\alpha_1v_1+...+\alpha_nv_n$ (perchè $\mathcal{B}=\{v_1,...,v_n\}$ genera $v$).
$T$ è lineare:
\[T(v)=T(\alpha_1v_1+...+\alpha_nv_n)=\alpha_1T(v_1)+...+\alpha_nT(v_n)\]
Questo determina univocamente $T$:
\\\textit{Devo dimostrare che se esiste un'altra applicazione lineare $S\neq T$ ma che coincide con $T$ sulla base, allora ho una contraddizione.}
Sia $S$ un'altra applicazione $S:V\rightarrow W$ con $T(v_i)=S(v_i)\;\forall i=1,...,n$ allora
\[S(v)=\alpha_1S(v_1)+...+\alpha_nS(v_n)=\alpha_1T(v_1)+...+\alpha_nT(v_n)=T(v)\]
\\
\\$T$ è determinata completamente dai suoi valori su una base $B$ di $V$.

\subsection{Corollario} 
Due applicazioni lineari coincidono $\Leftrightarrow$ coincidono sui vettori di una base.

\subsection{Corollario}
Se $\mathcal{B}=(v_1,...,v_n)$ base ordinata di $V$ e se $T:V\rightarrow W$ lineare, allora $ImT=Span(\{T(v_1),...,T(v_n)1\})$: cioè i vettori immagine della base di $V$ generano l'immagine della trasformazione
%TODO:COMPLETARE PG LEZ 32