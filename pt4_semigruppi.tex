\section{Semigruppo}
Sia X un insieme non vuoto. 
\\\textit{*}:
\[X\;*\;X\rightarrow Z\]
\[(a.b)\mapsto a * b\]
una operazione binaria associativa: \(\forall a,b,c\in X: a+(b+c)=(a+b)+c\)
\\
Un insieme \textit{X}, munito di una operazione associativa si chiama \textbf{semigruppo}.

\section{Monoide}
Se \((X,+)\) è un semigruppo e inoltre esiste un elemento \(1_X\) tale che \(a+1_X=1_X*a=a\) (\(1_X\) elemento neutro dell'operazione \textit{*}), allora \((X,+)\) si chiama \textbf{monoide}.

\section{Elenco gruppi}
\textbf{\((A^*, \cdot )\)} è un monoide non commutativo.
\\\textbf{\((\mathbb{N}, +) \)} (commutativo) monoide (0 el. neutro) ma non è un gruppo.
\\\textbf{\( (\mathbb{Z}, +)\)} gruppo commutativo (0 el. neutro).
\\\textbf{\((\mathbb{Q}, +) \)} gruppo commutativo (0 el. neutro); \(\frac{p}{a}\rightarrow\;opposto\;-\frac{p}{q}\).
\\\textbf{\((\mathbb{N}^*, \cdot)\)} monoide, non è un gruppo.
\\\textbf{\((\mathbb{Z}^*, \cdot)\)} monoide, non è un gruppo.
\\\textbf{\((\mathbb{Q}, \cdot)\)} non è un gruppo, 0 non ha inverso.
\\\textbf{\((\mathbb{Q}^*,\cdot)\)} gruppo.
\\\textbf{\((\mathbb{R}, +)\)} gruppo.
\\\textbf{\((\mathbb{R}^*, \cdot)\)} monoide, gruppo.
\\\textbf{\((\mathbb{Z}_n, +)\)} gruppo finito commutativo; el. neutro \(\overline{0}\).
\\\textbf{\((\mathbb{Z}_n,\cdot)\)} monoide, semigruppo (non è un gruppo \(\overline{0}\) non è invertibile.
\\\textbf{\((\cup (\mathbb{Z}_n),\cdot) \)} gruppo, el. neutro \(\overline{1}=\{\overline{a}: (a,n)=1\}\) (el. invertibili).
