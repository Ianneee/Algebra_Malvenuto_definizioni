\section{I numeri}
\subsection{Costruzione di \(\mathbb{Z}\) (interi)} a partire da \(\mathbb{N}\): prendiamo su \(\mathbb{N}\times\mathbb{N}\) la relazione \(\rho\) definita sulle coppie \((n, m)\in \mathbb{N}\times\mathbb{N}\) tale che \((n, m)\rho (n',m') \Leftrightarrow n+m'=m+n'\)

\subsection{Definizione di \(\mathbb{Z}\):} \[\mathbb{Z}=\mathbb{N}\times\mathbb{N}/_{\rho}\]

\subsection{Classi su \(\mathbb{Z}\):} 

\(\overline{(0,0)}\) zero

\(\overline{(m, 0)}, m>0\) positivi

\(\overline{(0, n)}, n>0\) negativi

\subsection{Sottoinsiemi di \(\mathbb{Z}\):} \[\mathbb{Z}=\mathbb{Z}^{>0}\cup\{0,0\}\cup\mathbb{Z}^{<0}\]

\subsection{Somma su \(\mathbb{Z}\):}
\[\overline{(n,m)}+\overline{(n',m')}=\overline{(n+n',m+m')}\]

\subsection{Prodotto su \(\mathbb{Z}\):}
\[\overline{(n,m)}\cdot\overline{n',m'}=\overline{(nn'+mm', nm'+mn')}\]

\subsection{Proprietà operazioni su \(\mathbb{Z}\):} \(\forall a,b,c\in\mathbb{Z}\) (coppie \(\overline{(n,m)}\)) valgono le seguenti:
\begin{enumerate}
    \item Associatività: \((a+b)+c=a+(b+c)\)
    \item Commutatività: \(a+b=b+a\)
    \item Esiste uno \textit{zero} per la somma, cioè un elemento \(0: a+0=0+a=a\)
    \item \(\forall a\in\mathbb{Z}\) esiste un elemento detto \textit{opposto}, denotato con \(-a\), cioè un elemento tale che: \(a+(-a)=(-a)+a=0\).
    
    \(a=\overline{(n,m)}\)
    
    \(-a=\overline{(m,n)}\)
    \item Associatività prodotto: \(a\cdot (b\cdot c)=(a\cdot b)\cdot c\)
    \item Commutatività prodotto: \(a\cdot b=b\cdot a\)
    \item Esiste un \textit{elemento neutro} per il prodotto, "1", cioè un numero in \(\mathbb{Z}\) tale che:
    \[a\cdot 1=1\cdot a=a\]
    \[\overline{(n,m)}\cdot\overline{(1,0}=\overline{(n,m)}\]
    \item Distributività del prodotto sulla somma:
    \[a\cdot (b+c)=a\cdot b+a\cdot c\]
\end{enumerate}

\subsection{Gruppo:} Un insieme S non vuoto, munito di una operazione \[m:S\times S\rightarrow S\]
\[(a,b)\mapsto m(a,b)=a\ast b\;\; (notazione\; infissa)\]
che verifica i punti 1, 3, 4 si chiama \(gruppo(S,\ast)\).
L'operazione su S è:
\begin{itemize}
    \item associativa
    \item con elemento neutro \textit{e}, \(\forall x, x\ast e=e\ast x=x\)
    \item per ogni elemento \textit{x} esiste un inverso rispetto al prodotto \(\ast\) cioè un elemento \textit{y} tale che \(x\ast y=y\ast x=e\), che si denota \(x^{-1}\)
\end{itemize}

\subsection{Gruppo commutativo (abeliano):} Se il gruppo \((S, \ast)\) soddisfa anche la proprietà 2 (quindi associatività, commutatività, elemento neutro, opposto).

\subsection{Anello:} Un anello è una terna \((A,+,\cdot)\) con:
\begin{itemize}
    \item A insieme non vuoto
    \item \(+ \; \cdot\) due operazioni binarie, associative
    \item \((A,+)\) è un gruppo abeliano
    \item Distributività: \(\forall a, b, c \in A, \; a\cdot (b+c)=a\cdot b+a\cdot c\)
\end{itemize}
\textbf{Anello commutativo:} Se un anello \((A,+,\cdot)\) il prodotto è commutativo, cioè se \(\forall a,b\in A,\;a\cdot b=b\cdot a\).

\subsection{Anello unitario:} Se esiste un elemento di A, che si denota con \(1_A\), tale che \(a\cdot 1_A=1_A\cdot a=a\).

\subsection{Divisore dello zero:} Un elemento \(a\in A,\; a\neq0_A\) di un anello di dice divisore dello zero se esiste \(b\in A,b\neq 0\) con \(a\cdot b=0_A\).

\subsection{Dominio di integrità:} se \((A,+,\cdot)\) è privo di divisori dello zero.

\subsection{Legge di annullamento del prodotto:} se in un dominio di integrità \(a\cdot b=0_A\) allora \(a=0_A\) oppure \(b=0_A\).

\subsection{Divisibilità:} dati \(a,b\in\mathbb{Z}\) si dice che a divide b, e si indica \(a|b\), se e solo se \(\exists c\in\mathbb{Z}\) tale che \(b=a\cdot c\).
La divisibilità è una relazione sugli interi:
\[a|b\Leftrightarrow\exists c\in\mathbb{Z}: b=a\cdot c\]

\subsection{Multiplo:} se \(a|b\) diremo che \textit{b} è un multiplo di \textit{a}.

\subsection{Associati}
\textit{a,b} sono associate se \(a|b\) e \(b|a\)
\newline\textit{Oss1:} in \(\mathbb{N^*}\) sono associati \(\Leftrightarrow a=b\).
\newline\textit{Oss2:} in generale, in \(\mathbb{Z}\Leftrightarrow a=b\) oppure \(a=-b\).

\subsection{Unità:}
In \(\mathbb{Z}\) sono +1 e -1.

\subsection{Irriducibile}
Un elemento \(a\in\mathbb{Z}, \; a\neq 0\) è irriducibile se \(a=b\cdot c\Rightarrow\) \textit{b} oppure \textit{c} sono unità.

\subsection{Primo:}
Un elemento \(a\in\mathbb{Z}\) si dice primo se:
\[a|b\cdot c\Rightarrow a|b \;oppure\; b|c\]

\subsubsection{Proposizione: in \(\mathbb{Z}\; a\) è primo \(\Rightarrow\;a\) irriducibile}
Sia \(a=b\cdot c\): usando l'ipotesi che a è primo allora \(a|b\) oppure \(a|c\).
\\
Se \(a|b \Rightarrow\exists\; h : b=a\cdot h \Rightarrow a = a\cdot h\cdot c \Rightarrow h\cdot c=1\Rightarrow c=\pm 1\)
\\
Allora \(a=b\cdot (+1)\) oppure \(a=b\cdot (-1)\), \textit{a} è irriducibile.

\subsubsection{Proposizione: in \(\mathbb{Z}\) a irriducibile\(\Rightarrow\)a primo}
Ipotesi: a irriducibile
\\
Tesi: a primo
Supponiamo che \(a|bc\Leftrightarrow\exists h\in\mathbb{Z}:bc=ah\),
\\
voglio mostrare che \(a|b\) oppure \(a|c\) ovvero che se \(a\nmid b\) allora \(a|c\).
\\
Ora a irriducibile, i suoi divisori sono \(a,-a,1,-1\). \(a\nmid b\) allora anche \(-a\nmid b\Rightarrow\)i divisori comuni tra \textit{a} e \textit{b} sono \(1,-1\rightarrow MCD(a,b)=1\).
\\\\
\centerline{\(\exists\)(id. Bézout)\(\exists\; h,k\in\mathbb{Z}\)}
\[1=ah+bk\]
\[c=cah+cbk=a(ck+k)\;\;\;[cb=a]\]
quindi \(a|c\).

\subsection{Massimo comune divisore:}
Dati \textit{a,b} non entrambi nulli, un elemento \(d\in\mathbb{Z}\) si chiama massimo comune divisore tra \textit{a} e \textit{b} un numero tale che:
\begin{itemize}
    \item \(d|a \land d|b\)
    \item Se \(c|a \land c|b\), allora \(c|d\): \textit{d} è il massimo tra i divisori comuni.
\end{itemize}
Chiamiamo massimo comune divisore l'unico positivo che soddisfa le due proprietà.

\subsubsection{Teor: Esistenza del MCD tra due numeri}
\(\forall a,b\in\mathbb{Z}\) non entrambi nulli, esiste un numero \(d\in\mathbb{N^*}\) tale che \(d=MCD(a,b)\)
\\
Il massimo comune divisore si esprime come una combinazione lineare tra \textit{a} e \textit{b}, ovvero esistono \(s, t\in\mathbb{Z}\) tali che \(d=s\cdot a+t\cdot b\) (\textit{identità di Bézout}).

Dimostrazione:
\\
Sia \(S=\{xa+yb:x,y\in\mathbb{Z}, xa+yb>0\}\)
\begin{enumerate}
    \item \(S\subseteq\mathbb{N}\)
    \item \(S\neq\emptyset\)
\end{enumerate}
a e b sono non entrambi nulli, quindi almeno uno dei due è \(\neq 0\).
Sia esso a.
\\
Se \(a>0\) allora \(1\cdot a+0\cdot b=a>0\)
Se \(a<0\) allora \((-1)\cdot a+0\cdot b=a>0\)
% Completare parte di dimostrazione dopo spiegazione

Dimostrazione che \(d|a\) e \(d|b\):
\\
Dividiamo \textit{a} per \textit{d} (divisione col resto):
\(\exists\; q,r\) con \(a=dq+r,\; 0\leq r<d\)
\\
Se \(r=0\) allora \(d|a\)
\\
Se \(r\neq 0\) allora \(0<r<d\)
\\
\(r=a-dq\); dato che \(d\in S\Rightarrow d=x_0a+y_0b\) allora\\ \(r=a-q(x_0a+y_0b)=a-qx_0a+qy_0b=a(1-qx_0)-(qy_0)b\)
\\
Quindi \(r\in S\) perchè è una combinazione lineare \(>0\) ma \(r<d\), però \textit{d} è il minimo di \(S\Rightarrow\)Assurdo.
\\\\
Dimostrazione se \(d'|a\) e \(d'|b\) allora \(d'|d\):
\\
Poichè \(d'|a\) e \(d'|b\) si ha che
\[\exists h:a=d'\cdot h, \exists k:b=d'\cdot k\]
Ora \[d=x_0a+y_0b\]
\[=x_0(d'h)+y_0(d'k)=\]
\[=d'(x_0h+y_0h)\Rightarrow d'|d\]

\subsubsection{Prop: se \(c|a\) e \(c|b\) allora \textit{c} divide ogni combinazione lineare di a e b}
\[a=ch\]
\[b=ck\]
\[\Rightarrow xa+yb=xch+yck\]
\[=c(xh+yk)\Rightarrow\in\mathbb{Z}\]
\[\Rightarrow c|xa+yb\]

\subsection{Proposizione}
\[1=at+bs\Rightarrow MCD(a,b)=1\]

\subsubsection{Lemma MCD(m,m+1)=1}
Sia \(m\in\mathbb{N},\; m\geq 1\) allora \(MCD(m,m+1)=1\).
\\\\
\textbf{Dimostrazione}: \[m+1-m=1\Rightarrow 1(m+1)+(-1)m=1\]
Potendo scrivere \textit{1} come combinazione lineare di \textit{m} e \textit{m+1}, \textit{m} e \textit{m+1} sono primi tra loro.

\subsection{Algoritmo di Euclide}

\subsubsection{Lemma1: L'algoritmo termina}

La successione dei resti è un numero \(0\leq ... <r_2<r_1<b\).

\subsubsection{Lemma2: Se \(a=bq+r\) \(MCD(a,b)=MCD(b,r)\)}
% Dimostrazione da completare dopo chiarimento
TODO: scrivere dimostrazione

\subsubsection{Corollario: \(MCD(a,b)=MCD(r_n,0)=r_n1\)}
Per il lemma 2 \(MCD(a,b)=MCD(b,r_1)=MCD(r_1,r_2)=...=MCD(r_{n-1},r_n)=MCD(r_n,0)\)

\subsubsection{Lemma3}
Se \(x\in\mathbb{N^*}\) allora \(MCD(x,0)=x\)

\subsection{Coprimi}
\textit{a,b} non entrambi nulli, \textit{a} e \textit{b} si dicono coprimi (o \textit{primi fra loro}) se \textit{MCD(a,b)=1}.

\subsubsection{Osservazione1}
Se \textit{a} e \textit{b} sono primi fra loro, allora \[\exists\; x,y\in\mathbb{Z} : 1=xa+yb\]

\subsubsection{Osservazione 2}
Se \[d=MCD(a,b)\Rightarrow\exists\; x,y:d=ax+by\]

\subsubsection{Proposizione 1}
Se \(\exists\; x_0,y_0\) con \(1=ax+by\) allora \textit{a,b} sono primi tra loro.

\subsubsection{Proposizione 2}
Se \textit{a} e \textit{b} sono coprimi e dividono un terzo numero \textit{c}, allora \(ab|c\).

\subsection{Equazione diofantea}
Equazione con una o più incognite sugli interi di cui si cercano le soluzioni intere.

\subsubsection{Teor: Soluzione equazione diofantea}
L'equazione diofante lineare in \textit{x} e \textit{y} \(ax+by=c\;\; a,b,c\in\mathbb{Z}\) possiede soluzioni intere \((x,y)\in\mathbb{Z}^2\Leftrightarrow d=MCD(a,b)|c\)
\\
(Dim\(\Rightarrow\)) La condizione \(MCD(a,b)|c\) è necessaria.
\\
Ipotesi: esiste una soluzione di \(x^2+y^2=z^2\)
\\
Tesi: \(d|\)termine noto, \(d=MCD(a,b): d|a\) e \(d|b\Rightarrow d|\) ogni combinazione lineare di \textit{a,b}.
\\
Se \(x_0,y_0\) sono una soluzione, allora \(ax_0+by_0=c\Rightarrow d|c=ax_0+by_0\)
\\\\
(Dim\(\Leftarrow\)) La condizione è sufficiente.
\\
Ipotesi \(MCD(a,b)=ah+bk\), per opportuni \(h,k\in\mathbb{Z}\)

\subsection{Teorema fondamentale dell'aritmetica}
\(\forall n>1, n\in\mathbb{N},\exists\;p_1,...,p_j\in\mathbb{N}\) (irriducibili) \(\exists h_1,...,h_j\geq 1\) tali che:
\begin{itemize}
    \item \(n=p_1^{h_1}...p_j^{h_j}\;\;p_1,...p_j\) distinti
    \item la fattorizzazione di \(n=p_1^{h_1}...p_j^{h_j}\;\;p_1,...p_j\) è unica a meno di riordinare i fattori
\end{itemize}

\subsubsection{Osservazione 1}
j può essere 1, cioè potrebbe esserci un solo irriducibile nella fattorizzazione di \textit{n}, anche \textit{h} possono essere 1.
Se \textit{n} è irriducibile \(\Rightarrow n=n\) è la fattorizzazione in irriducibili di \textit{n}.

\subsubsection{Osservazione 2}
1 non è considerato irriducibile perché si perderebbe l'unicità della scrittura in irriducibili.

\subsubsection{Dimostrazione esistenza}
Con principio di induzione in forma forte.
\\
\textbf{Base}: \textit{n=2}, 2 è irriducibile.
\\
Per \textbf{oss1} \(2=2^1\) è la fattorizzazione in primi in irriducibili di 2
\\
\textbf{Ipotesi induttiva}: ogni \(2\leq a<n\;\;(2\leq a\leq n-1)\) è fattorizzabile in irriducibili: \(\exists\; \alpha _1...\alpha _t \alpha _i\leq 1\) e \(q_1,...q_t\) irriducibili con \(a=q^{\alpha _1}_1...q_t^{\alpha _t}\)
\\
\textbf{Passo induttivo}: provare che \textit{n} sia prodotto di irriducibili
\\\\
\textbf{Primo caso}: \textit{n} irriducibile \(\rightarrow\) fatto, per \textit{oss.1}
\\\\
\textbf{Secondo caso}: \textit{n} riducibile: \(\exists\;b,c\in\mathbb{Z}, 1\neq b, c\neq n\) (divisori propri) con \(n=bc\Rightarrow 2\leq b,c<n\).
\\
Allora per \textit{b} e \textit{c} vale l'ipotesi induttiva e quindi
\[b=q_1^{\alpha _1}...q_t^{\alpha _t}\;\; c= x_1^{\beta _1}...x_s^{\beta _s}\]
\[n=bc=q_1^{\alpha _1}...q_t^{\alpha _t} x_1^{\beta _1}...x_s^{\beta _s}\]

\subsection{Dimostrazione unicità}
Per induzione su \textit{m}, con \textit{m} è la lunghezza minima di una fattorizzazione per \textit{n}.
\\
\textit{m}: minimo numero di irriducibili di una fattorizzazione di \textit{n}
\\
\textbf{Base}: \(m=1\Rightarrow n=n\) è primo.
\\
Se per assurdo \(n=q_1...q_s,\; s\geq 2\) allora \(n|q_1\) o \(n|q_2...q_s\).
\\
Prendiamo \(n|q_1\), anche \(q_1\) è primo \(\Rightarrow n=q_1\); semplificando da entrambe le parti \(\Rightarrow 1=q_2....q_s\) che porterebbe ad un assurdo perché \(1=1\).
\\
Quindi \(n=q_1\) ed è l'unica fattorizzazione.
\\\\
\textbf{Ipotesi induttiva}: se il minimo numero di primi in una fattorizzazione di \textit{n} è \(m-1\), allora la fattorizzazione è unica a meno dell'ordine.
\\\\
\textbf{Passo induttivo}: \textit{m} è il minimo di una fattorizzazione di \textit{n}.

\subsection{Teor. Euclide - Esistenza infiniti primi}
L'insieme \(P=\{p\in\mathbb{N} :\) p è primo\(\}\) è infinito.
\\
\textbf{Dimostrazione}: Supponiamo che \textit{P} sia finito, cioè \(P=\{p_1,...,p_n\}\).
\\
Sia \(m=p_1,...p_n\) il prodotto di tutti i primi.
\\
Considero \(m+1\): per il teorema fondamentale dell'aritmetica \(m+1=p_1^{k_1}...p_n^{k_n}\), \(k_1,...,k_n\geq 0\) almeno uno degli esponenti \textit{>0}.
\\
Per il lemma su MCD di un numero ed il suo successivo \textit{m} e \textit{m+1} sono coprimi.
\\
Sia \textit{j} tale che \(k_j>0\), cioè \(p_j^{k_k}|m+1\); vale anche \(p_j|m\) allora \(p_j|MCD(m, m+1)=1\) che è un assurdo.
