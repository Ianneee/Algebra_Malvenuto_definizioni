\section{Congruenze}

\subsection{Congruenza modulo n}
La congruenza modulo n (n fissato) è una relazione di equivalenza definita su \(\mathbb{Z}\).

\(x\equiv y(mod\; n)\Leftrightarrow x-y\) multiplo di \textit{n} \(\Leftrightarrow n|x-y\)

\subsection{Proposizione}
La congruenza \textit{(mod n)} è una relazione di equivalenza.
\\
\textbf{Dimostrazione}:
\\\\
\textit{(R)} \[\forall x\in\mathbb{Z}: x\equiv x(mod\; n)\Leftrightarrow n|(x-x)\]
Vera perché \(0=0\cdot n\).
\\\\
\textit{(S)}
\[\forall x,y\in\mathbb{Z}: x\equiv y(mod\; n)\Rightarrow y\equiv x(mod\; n)\]
So che \(n|x-y\Leftrightarrow x-y=nh\) per qualche \(h\in\mathbb{Z}\).
\\\\
Moltiplicando per \textit{-1}: \(y-x=-nh=n(-h)\) quindi \(n|y-x\Rightarrow y\equiv x(mod\;n)\)
\\\\
\textit{(T)}
\[x\equiv y(mod\;n)\land y\equiv z(mod\;n)\Rightarrow x\equiv z(mod\;n)\]
\((x-y)=nh_1 \land (y-z)=nh_2\)
\\
\((x-z)=(x-y)-(y-z)=nh_1-nh_2=n(h_1-h_2)\) quindi \(n|x-z\Rightarrow x\equiv z(mod\;n)\)

\subsection{Quoziente}
Il quoziente della congruenza \textit{(mod n)} si denota come \(\mathbb{Z}_{/ \equiv (mod\;n)}=\{[x]_n:x\in\mathbb{Z}\}\).
\\
Il quoziente \(\mathbb{Z}_n\) si chiama anche \textbf{interi modulo n}.

\subsection{Proposizione - Resto}
Dati \(x,y\in\mathbb{Z}\) si ha: \(x\equiv y(mod\;n)\Leftrightarrow\) il resto delle divisioni di \textit{x} e di \textit{y} per \textit{n} è lo stesso.
\\
\textbf{Dimostrazione \(\Rightarrow\)(se \(x\equiv _n y\) hanno lo stesso resto}
\(x-y=nh\) (per qualche h)\\
\(x=nh+y\)\\
Dividendo \textit{y} per \textit{n}: \(\exists !q,r\in\mathbb{Z} : y=nq+r,\; 0\leq r<n\).
\\
Scambiando in \textit{x}: \(x=nh+nq+r=n(h+q)+r\), \textit{x} ed \textit{y} hanno quindi lo stesso resto.

\subsection{Osservazione}
Sia \(x=nq+r,\;0\leq r<n\) la divisione con resto di \textit{x} per \textit{n}.\\
Allora \[[x]_n=[r]_n\Leftrightarrow x\equiv r(mod\;n)\Leftrightarrow x-r=nq\]
Quindi \[n|x-r\]

\subsection{Proposizione somma}
La somma classi resto in \(\mathbb{Z}_n\), definita da: \(\overline{x}+\overline{y}:= \overline{x+y}\), è ben posta, ovvero non dipende dalla scelta dei rappresentanti.
\\
\textbf{Dimostrazione}
Siano \(x'\in\overline{x}\), cioè \(\overline{x'}=\overline{x}\) e \(y'\in\overline{y}\) cioè \(\overline{y'}=\overline{y}\), allora\\
\[x'\equiv x(mod\;n)\Leftrightarrow x'=x+kn\]
\[y'\equiv y(mod\;n)\Leftrightarrow y'=y+hn\]
Da verificare: \(\overline{x'+y'}=\overline{x+y}\Leftrightarrow x'+y'=x+y+tn\)
\\Quindi:
\[x'+y'=x+kn+y+hn\]
\[=x+y+kn+hn\]
\[=x+y+(k+h)n\;\;[(k+h)=t]\]

\subsection{Dimostrazione prodotto}
\[x'\cdot y'= (x+kn)(y+hn)\]
\[=xy+xhn+kny+khn^2\]
\[xy+n(xh+ky+khn),\;\;\;[(xh+ky+khn)=t]\]

% \subsection{Campo}
% Un campo è una terna \((K,+,\cdot)\) con \textit{K} insieme non vuoto e 2 operazioni.
% \begin{itemize}
%     \item \((K,+,\cdot)\) anello commutativo unitario
%     \item Detto \(0_k\) l'elemento neutro della somma e denotato con \(K^*=K\setminus\{0_k\}\), deve valere che \(\forall x\in K^*:x\cdot x^{-1}=1_k\)
% \end{itemize}
% Quindi campo \(\Leftrightarrow\) anello commutativo unitario con in più \(K\setminus\{0_k\}=(K^*,\cdot)\) gruppo.

\subsection{Proposizione - Invertibilità}
\(a\in\mathbb{Z}, \overline{a}\) invertibile in \(\mathbb{Z}_n\Leftrightarrow MCD(a,n)=1\)
\\
\textbf{Dim \(\Rightarrow\)}
\\Ipotesi: \(\overline{a}\in\mathbb{Z}\) invertibile
\\\\Tesi: \textit{(a,n)=1}
\\\\Esiste \(b\in\mathbb{Z}: \overline{a}\cdot\overline{b}=1\) 
\[\Leftrightarrow ab\equiv 1(mod\;n)\]
\[\Leftrightarrow n|1-ab\]
\[\Leftrightarrow  1-ab=nk\]
\[\Leftrightarrow 1=ab+nk\]
\[\Rightarrow MCD(a,n)=1 \]
\\
\textbf{Dim \(\Leftarrow\)} 
\\Ipotesi: \(MCD(a,n)=1\)
\\\\Tesi: \(\overline{a}\) è invertibile
\\\\Se \(MCD(a,n)=1\) allora esistono \(h,k\in\mathbb{Z}:\)
\[1=ah+nk\;\;\in\mathbb{Z}\]
\[\overline{1}=\overline{ah+nk}\]
\[\overline{1}=\overline{a}\overline{h}+\overline{n}\overline{k}\;\;\in\mathbb{Z}\]
\[\overline{n}\overline{k}=\overline{0}\overline{k}\]
\[\overline{1}=\overline{a}\overline{h}\Rightarrow\overline{h}=(\overline{a})^{-1}\]

\subsection{Classi resto invertibili}
\[\cup (\mathbb{Z}_n):=\{a\in\mathbb{Z}_n:\overline{a}\;invertibile\}\subseteq\mathbb{Z}_n\]
\[\cup (\mathbb{Z}_n)=\{\overline{a}: MCD(a,n)=1\}\]

\subsection{Teorema Uguaglianza sbagliata}
Se p è primo allora \(\forall x,y\in\mathbb{Z}\) vale:
\[(x+y)^p\equiv x^p+y^p(mod\;p)\]
\[(\overline{x}+\overline{y})^p=\overline{x}^p+\overline{y}^p(mod\;p)\]
\textbf{Dimostrazione:}\((x+y)^p=\sum ^p_{i=0}\binom{p}{i}x^iy^{p-i}\)
\[\binom{p}{0}=1=\binom{p}{p}\]
\[\binom{p}{0}x^0y^p=1y^p\]
\[\binom{p}{p}x^py^0=1x^p\]
\\
Considerare con \(0<i<p\) il coefficiente binomiale è:
\[\binom{p}{i}=\frac{p(p-1)...(p-i+1)}{i(i-1)...2\cdot 1}\in\mathbb{N}\]
\[p(\frac{(p-1)...(p-i+1)}{i!})\Rightarrow p|\binom{p}{i}\forall i=2,...,p-1\]
\[\Rightarrow\binom{p}{i}\equiv 0(mod\;p)\]

\subsubsection{Grande teorema di Fermat}
\(x^n+y^n=z^n, n\geq 3\) non ha soluzioni intere.

\subsubsection{Piccolo teorema di Fermat}
\(\forall a\in\mathbb{Z}, \forall p(mod)\) primo si ha che: \(a^p\equiv a(mod\;p)\) in \(\mathbb{Z_p}\), \textit{p} primo vale \(\overline{a}^p=\overline{a}\).
\\\\
\textbf{Dimostrazione per \(a\in\mathbb{N}\)}
\\Per induzione su \textit{a}
\\\\\textbf{Base}: \[a=0\]
\[0^p\equiv ^? 0(mod\;p)\]
\[0^p=0\in\mathbb{Z}\Rightarrow 0^p\equiv (mod\;p)\]
\\\\
\textbf{Ipotesi induttiva:} supponiamo vera per \textit{a} l'affermazione \(a^p\equiv a(mod\;p)\)
\\\\
\textbf{Passo induttivo:} verifichiamo per \((a+1)\).
\[(a+1)^p\equiv a^p+1^p \equiv a+1\]
\(a^p\rightarrow a\) e \(1^p\rightarrow 1\) per ipotesi induttiva.
\\\\
Se \(a<0\) è ancora vero?
\\ Se \(a<0\) allora \(-a>0\), cioè \((-a)^p\equiv -a(mod\;p)\). 
Ora:
\[0=a-a\]
\[0^p=(a-a)^p\]
\[0^p\equiv (a-a)^p\equiv a^p+(-a)^p\]
\[\equiv a^p-a\equiv 0\cdot (mod\;p)\Leftrightarrow a^p\equiv a(mod\;p)\]

\subsection{Teorema Eulero-Fermat}
Se \((a,p)=1\) cioè se \(\overline{a}\neq \overline{0}\) in \(\mathbb{Z}_p\) allora
\[a^{p-1}\equiv 1(mod\;p)\]
\\\\
\textbf{Dimostrazione:} se \((a,p)=1\) allora esiste l'inverso moltiplicativo di \(\overline{a}\) in \(\mathbb{Z}_p\).
\\
So che \[a^p\equiv a(mod\;p)\]
\[(\overline{a}^p)\equiv \overline{a}(mod\;p)\]
\[\Rightarrow moltiplicando\;per\;l'inverso \Rightarrow \overline{a}^{p-1}=\overline{1}\;in\;\mathbb{Z}_p\]
\[\Leftrightarrow a^{p-1}\equiv 1(mod\;p)\]

\subsection{Corollario}
Se \((a,p)=1\) e se \textit{p} primo allora \(\overline{a}^{p-2}\) è l'inverso moltiplicativo di \(\overline{a}\) in \(\mathbb{Z}_p\)
\\\\
\textbf{Dimostrazione:} l'inverso di \(\overline{a}\) è \(\overline{x}\) con \(\overline{a}\cdot\overline{x}=\overline{2}\), 
ma \[\overline{a}\cdot\overline{a}^{p-2}=\overline{a}^{p-1}=\overline{1}\]
per il \textit{teorema di Eulero-Fermat}.
