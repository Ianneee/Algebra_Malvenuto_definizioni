\section{Spazi Vettoriali}

\subsection{Definizione spazio vettoriale}
Uno spazio vettoriale \(V\) su un campo \(K\) è

\begin{itemize}

	\item Un insieme nn vuoto \(V\), in cui sono definite due operazioni, di cui una interna ed una esterna.
	\\\textbf{Interna:} somma \(+\):
	\[V\times V\rightarrow V\]
	\[(v,w)\mapsto v+w\]
	\textbf{Esterna:} prodotto \(\cdot\) per uno scalare:
	\[K\times V\rightarrow V\]
	\[(c, v)\mapsto c\cdot v\]

	\item \((V,+)\) è un gruppo commutativo

	\item \(K\times V\rightarrow V\) e \((c, v)\mapsto c\cdot v\) tale che:

	\begin{itemize}

		\item distributività per vettori:
		\[\forall c\in K,\forall v,w\in V\]
		\[c(v+w)=cv+cw\]
		\[(v+w)c=vc+wc\]

		\item associatività per gli scalari:
		\[\forall c,d\in K, \forall v\in V\]
		\[c(dv)=(cd)v\]
		
	\end{itemize}

	\item distributività per gli scalari:
	\[\forall c,d\in K, v\in V\]
	\[(c+d)v=cv+dv\]

	\item \(1\cdot v=v\)


\end{itemize}

\subsection{Scalare}
E' un elemento del campo.

\subsection{Sottospazio vettoriale}
Un sottospazio vettoriale di uno spazio vettoriale \(V\) su \(R\) è un sottoinsieme \(W\subseteq V\) non vuoto tale che: \(W\) rispetto le stesse operazioni di \(V\) sia esso stesso uno spazio vettoriale.
\\
\begin{itemize}

	\item Equivalentemente si deve avere:

	\begin{enumerate}

		\item \((W,+)\) è un sotto gruppo di \((V,+)\)

		\item chiuso rispetto alla moltiplicazione per uno scalare

	\end{enumerate}

	\item Equivalentemente:

	\begin{enumerate}

		\item \(\forall\;u,v\in W\): \(u-v\in W\)  [\(a\cdot b^{-1}\in G\)]

		\item \(\forall\;\alpha\in\mathbb{R},v\in W\): \(\alpha\cdot v\in W\)

	\end{enumerate}	

	\item Equivalentemente:

	\begin{enumerate}

		\item \(\forall\; u,v\in W: u-v\in W\)

		\item \(\forall\alpha\in\mathbb{R},\forall v\in W: \alpha v\in W\)
		\\(Se \(\alpha= -1,v\in W\) allora \(-v\in W\)
		\\\(u\in W\) allora \(u-(-v)=u+v\))
		
	\end{enumerate}

	\item Equivalentemente \(W\subseteq V\) è un sottospazio vettoriale \(\Leftrightarrow\)

	\begin{enumerate}

		\item [1*] \(\forall\;u,v\in W: u+v\in W\)

		\item [2*] \(\forall\;\alpha\in\mathbb{R},\forall v\in W: \alpha v\in W\)
		
	\end{enumerate}
	
\end{itemize}

\subsection{Proposizione}

\(W\subseteq V,W\neq\emptyset\) è un sottospazio vettoriale \(\Leftrightarrow\):
\[\forall\;\alpha\beta\in\mathbb{R},\forall\; u,v\in W:\; \alpha u+\beta v\in W\]
\(\alpha u+\beta v\) si chiama \textbf{combinazione lineare} di \(u\) e \(v\).
\\
\\\textbf{Dimostrazione:} la combinazione lineare è equivalente a \(1*\) e \(2*\).
Supponiamo che \(\alpha u+\beta v\in W\) \(\forall\;\alpha\beta\in\mathbb{R},\forall\; u,v\in W\)
\\\(2*\rightarrow\) in particolare è vero se prendo \(\alpha =\alpha, \beta=0\): \(\alpha u+\beta v=\alpha u\in W\).
\\\(1*\rightarrow\) in particolare, se prendo \(\alpha =1,\beta =-1\): so che \(1\cdot u+(-1)\cdot v= u-v\in W\).

\subsection{Definizione: traccia}
Data una matrice quadrata \(A=[a_{i,j}]\) si chiama traccia della matrice il valore (scalare in \(\mathbb{R}\)) definito da:
\[tr(A)=a_{11},a_{22},a_{33}+...+a_{nn}\]
(è la somma degli elementi della diagonale).

\subsection{Definizione: combinazione lineare}

Dati \(v_1,...v_t\in V\) vettori di uno spazio vettoriale \(V\) su \(\mathbb{R}\) dati \(t\) scalari \(c_1,c_2,...,c_t\in\mathbb{R}\), il vettore \(v=c_1v_1+c_2v_2+...+c_tv_t\) si chiama combinazione lineare di vettori \(v_1,...,v_t\) tramite gli scalari \(c_1,...,c_t\).

\subsection{Proprietà di calcolo negli spazi vettoriali}
\(V\) spazio vettoriale su \(K\), \(0\) è lo zero del campo, \(0_V=\underline{0}\) è l'elemento neutro del gruppo \((V,t)\)

\begin{itemize}

	\item \(0v=\underline{0}\) vettore nullo \(\forall v\in V\)

	\item \((-c)v=-(cv)\) \(\forall\;v\in V,\forall c\in\mathbb{R}\)

	\item \(c\underline{0}=\underline{0}\)

	\item Se \(cv=\underline{0}\) allora \(c=0\) oppure \(v=\underline{0}\)

\end{itemize}

\subsection{Definizione}

\(w\) è combinazione lineare di \(v_1,v_2,...,v_t\) se esistono degli scalari \(c_1,c_2,...,c_t\in\mathbb{R}\) tali che:
\[w=c_1v_1+...+c_tv_t\]

\subsection{Osservazione: combinazione lineare banale}
Lo "zero" vettoriale è sempre combinazione lineare di un insieme \(\{v_1,...,v_t\}\) di vettori qualunque:
\[\underline{0}=0v_1+0v_2+...+0v_t\]

\subsection{Definzione: linearmente dipendente}
Un insieme di vettori \(\{v_1,...,v_n\}\) è linearmente dipendente (sul campo di \(V)\Leftrightarrow\) esistono coefficienti \(c_1,...,c_n\in K\) non tutti nulli, tali che:
\[c_1v_1+c_2v_2+...+c_nv_n=\underline{0}\]

\subsection{Osservazione}
Se almeno uno dei coefficienti \(\{c_1,...,c_n\}\) è non nullo (sia \(C_j\neq 0\)), allora si può scrivere (partendo dalla precedente \textit{linearmente dipendente}):
\[c_jv_j=-c_1v_1-c_2v_2-...-c_{j-1}v_{j-1}-c_{j+1}v_{j+1}-....-c_nv_n\]
e \(C_j\neq 0\Rightarrow \exists c_j^{-1}\) allora:
\[v_j=-\frac{c_1v_1}{c_j}-\frac{c_2v_2}{c_j}-...-\frac{c_nv_n}{c_j}\]

\subsection{Osservazione: linearmente indipendente}
\(\{v_1,...v_n\}\) è un insieme linearmente \textbf{indipendente} \(\Leftrightarrow\) \(\underline{0}=c_1v_1+...+c_nv_n\Leftrightarrow c_1=c_2=...=c_n=0\).

Ovvero: \(\{v_1,...,v_n\}\) sono vettori linearmente indipendenti \(\Leftrightarrow\) l'unica combinazione lineare di \(v_1,...,v_n\) è la combinazione lineare banale.

\subsection{Osservazione}

Il vettore nullo \(\underline{0}\) di \(V\) è sempre linearmente dipendente da qualunque insieme finito di vettori.

Infatti sia \(\{u_1,...u_z\}\subseteq V\) allora:
\[\underline{0}=1\cdot\underline{0}=0u_1+0u_2+...+0u_t\]
(un modo equivalente: \(0u_1+0u_2+...+0u_t-1\cdot\underline{0}=\underline{0}\))

\subsection{Osservazione}
Se \(S\subseteq V\) con \(\underline{0}\in S\), \(S=\{\underline{0},v_1,...,v_k\}\) allora \(S\) è un insieme di vettori dipendenti: infatti c'è la dipendenza
\[1\cdot\underline{0}=0v_1+0v_2+...+0v_k\]

\subsection{Osservazione}

La proprietà di essere indipendente di \(S\subseteq V,\) \(S=\{v_1,...,v_t\}\) si eredita ai sottoinsiemi, cioè:
\[\forall\; T\subseteq S,\;S\;indipendente\Rightarrow T\;indipendente\]
\(\{v_1\}\) è un insieme indipendente \(\Leftrightarrow v_1\neq 0\); \(\{\underline{0}\}\) è indipendente.

\subsection{Sottospazio generato da: span}
Dati \(v_1,v_2,...,v_t\) vettore di \(V\) (spazio vettoriale su un campo) lo \textit{span} dei vettori \(v_1,...,v_t\) è il più piccolo sottospazio vettoriale di \(V\) che contiene \(v_1,...v_t\)
\[Span(v_1,...,v_t)=\bigcap _{W\leq V\;\{v_1...v_t\}\in W} W\]
\textit{Altra notazione Span}: \(<v_1,...v_t>\)

\subsection{Proposizione}
\[Span(v_1,...v_t)=\{\sum _{i=1} ^t\alpha _iv_i:\alpha _i,...,\alpha _t\in K\}\]
Dimostrazione data per esercizio

\subsection{Sistema di generatori}
Dato \(V\) su \(K\) (es. \(K=\mathbb{R})\), i vettori \(\{v_1,...v_t\}\) sono un sistema di generatori (o insieme di generatori) per \(V\) se 
\[V=Span(v_1,...,v_t)\]
\\
\\Se \(W\subseteq V\) è un sottospazio, allora \(\{u_1,...,u_k\}\) sono generatori (sistema di generatori) per \(W\) se
\[W=Span(u_1,...,u_k)\]

\subsection{Basi di spazi vettoriali}
Dato \(V\) su \(K\), un insieme \(B=\{v_1,...,v_n\}\subseteq V\) si chiama base di \(V\) se:
\begin{itemize}

	\item \(V=Span(v_1,...,v_n)\) cioè \(B\) sono generatori per \(V\).

	\item \(\{v_1,...,v_n\}\) sono indipendenti.

\end{itemize}

\subsection{Spazio finitamente generato}
Uno spazio vettoriale $V$(su $K$), si dice finitamente generato se ammette un insieme finito di generatori.

\subsection{Proposizione}
$B=\{v_1,...,v_n\}$ é una base di $V\Leftrightarrow \forall v\in V\exists !\;(c_1,...,c_n),c_i\in\mathbb{R}$ con \\$v=c_1v_1+c_2v_2+...+c_nv_n$
\\
\\\textbf{Dimostrazione$\Rightarrow$}
\begin{enumerate}
	\item Ipotesi: $B=\{v_1,...,v_n\}$ é una base di $V$
	\item Tesi: $v=c_1v_1+c_2v_2+...+c_nv_n$
\end{enumerate}
Sia $v\in V$. Siccome $B$ è una base, allora è un insieme di generatori di $V$
\[\Rightarrow v\in Span(v_1,...,v_n)\Rightarrow \exists c_1,...,c_n \text{ con } v=c_1v_1+...+c_nv_n\]
 \\Perchè sono unici? Siano $(d_1,...,d_n)$ con 
 \[v=d_1v_1+...+d_nv_n\]
 \[v=c_1v_1+...+c_nv_n\]
 \[\Rightarrow v-v=0=d_1v_1+...+d_nv_n-(c_1v_1+...+c_nv_n)=\]
 \[=(d_1-c_1)v_1+...+(d_n-c_n)v_n\]
 ma $v_1,...,v_n$ sono indipendenti $\Rightarrow$ i coefficienti $(d_kv_k)$ sono tutti nulli.
 \\
 \\\textit{Dimostrazione $\Leftarrow$ fare per esercizio}

 \subsection{N-upla delle coordinate di $v$ in base $B$}
 Data $B=\{v_1,...,v_n\}$ una base ordinata di V, se $v=c_1v_1+...+c_nv_n$ allora il vettore colonna $\begin{bmatrix}c_1\\\vdots\\c_n\end{bmatrix}\in\mathbb{R}^n$ si chiama vettore delle coordinate di $v$ in base $B$.

 \subsection{Corollario}
 Fissata una base $B=\{v_1,...,v_n\}$ di $V$ l'applicazione $\varphi : V\rightarrow K^n$ ($K$ è il campo di $V$) che manda $v$ in $\varphi (v)=\begin{bmatrix}c_1\\\vdots\\c_n\end{bmatrix}$ (cioè il vettore nelle sue coordinate in base $B$) è una biiezione.

 \subsection{Teorema: esistenza di una base}
 Se $V$ è uno spazio vettoriale finitamente generato (cioè se $V=Span(v_1,...,v_h)$) allora esiste una base $\{w_1,...,w_n\}$ di $V$.
 \\\textit{Non dimostrato.}

 \subsection{Proposizione}
 Date due basi $B=\{v_1,...,v_n\}$ e $B=\{w_1,...,w_m\}$ di $V$ allora $n=m$.

 \subsection{Dimensione di $V$}
 Se $V$ è finitamente generato, allora si chiama dimensione di $V$ la cardinalità di una qualunque base di $V$.

 \subsection{Osservazione (notazione)}
 Se la dimensione di $V$ è $n$, allora $\varphi :V\rightarrow K^n$ si scrive $dim_KV=n$ se $|B|=n$.

 \subsection{Teorema del completamento di una base}
 Sia $B=\{v_1,...,v_n\}$ una base di $V$, e sia $\{w_1,...,w_p\}$ con $\{p\leq n\}$ un insieme di vettori indipendenti. Allora posso completare $\{w_1,...,w_p\}$ con $(n-p)$ vettori di $B$ a formare una base (ovvero posso sostituire $p$ vettori di $B$ un altro insieme di $p$ vettori indipendenti).
 \\\textit{Dimostrazione fatta parzialmente da vedere sul libro se si vuole la lode.}

 \subsection{Teorema}
 Le seguenti condizioni sono equivalenti tra loro per un insieme $B=\{v_1,...,v_n\}\subseteq V$ di $n$ vettori:
 \begin{enumerate}
 	\item $B$ è una base.
 	\item $B$ è un insieme di generatori minimale (cioè ogni sottoinsieme $S$ proprio di $V$ genera un sottospazio $Span(S)\not\leq V$)
 	\item $B$ è un insieme di vettori linearmente indipendenti massimale (cioè ogni $T\not\geq B$ insieme di vettori che contiene $B$ non è più indipendente).
 \end{enumerate}

 \subsection{Corollario}
 Se $V$ è finitamente generato e $B=\{v_1,...,v_n\}\;\;B'=\{v'_1,...,v_m\}$ sono basi di $V$ allora $|B|=|B'|$ $(n=m)$.
 \\
 \\\textbf{Dimostrazione:} per assurdo $m<n$, allora con il teorema del completamento costruisco $B''=B'\cup\{n-m\;vettori\}$. $B'$ è una base, ma per il pt.3 del teorema precedente, $B'$ è un insieme massimale $\Rightarrow$ contraddizione.

\subsection{Osservazioni}
\begin{itemize}
	\item Se $dim_KV=n$ allora ogni sottoinsieme di $n$ vettori indipendenti è anche un insieme di generatori.
	\item Se $dim_KV=n$ allora ogni insieme di $n$ che generano $V$ è anche un insieme indipendente (cioè una base).
	\item Se $W\leq V$, sottospazio di $V$ con $dim_KV=n$ allora:
	\begin{itemize}
		\item $dim_KW\leq dim_KV$
		\item $dim_KW=dim_RV\Leftrightarrow W=V$
		\item $W\neq\{0_V\}\Leftrightarrow dim_KW>0$
	\end{itemize}
\end{itemize}

\subsection{Somma di sottospazi}
Dati $U,W\leq V$ ($V$ spazio vettoriale su $R$):
\begin{enumerate}
	\item L'intersezione $U\cap W$ è anche un sottospazio di $V$
	\[U\cap W\{v\in V: v\in U\land v\in W\}\]
	\textit{Dimostrazione per esercizio}
	\item (L'unione di due sottospazi non è, in generale, un sottospazio (a meno che non siano uno contenuto nell'altro)).
	Il più piccolo sottospazio di $V$ che contiene sia $V$ che $W$ si chiama la \textbf{somma} di $U$ e $W$ e si denota:
	\[U+W=Span(U\cup W)\]
\end{enumerate}

\subsection{Proposizione}
\[U+W=\{u+w: u\in U,w\in W\}\]
\textit{Per esercizio verificare che u+w è un sottospazio che contiene $U\cup W\;e\;Span(U+W)$}

\subsection{Teorema di Grossman}
$dim_K(U+V)=dim_KU+dim_KW-dim_K(U\cap W)$

\subsection{Somma diretta di sottospazi}
Quando $U\cap V=\{0_V\}$, cioè ha dimensione $0$, allora si parla di somma diretta di sottospazi e si scrive:
\[U\oplus W\]

\subsection{Proposizione}
Se $U\cap V=\{0_V\}$ allora ogni vettore di $U\oplus W$ si scrive come somma di un elemento di $U$ più un elemento di $W$ in modo unico.
\\
\\\textbf{Dimostrazione:} supponiamo che $v_0\in U+W$ si scriva in due modi diversi:
\[v_0=u+w=u'+w'\;\;(con\;u,u'\in U;\;w,w'\in W)\]
\[\Rightarrow v_0-v_0=0=u+w-(u'+w')=\]
\[(u-u')+(w-w')=0\]
\[\Rightarrow \in U\cap W=\{0_V\}\]
\[\Rightarrow u-u'=0\Rightarrow u=u'\]
\[\Rightarrow w-w'=0\Rightarrow w=w'\]
