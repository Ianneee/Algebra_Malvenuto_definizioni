\section{Strutture algebriche}

\subsection{Gruppo} 
Un insieme S non vuoto, munito di una operazione \[m:S\times S\rightarrow S\]
\[(a,b)\mapsto m(a,b)=a\ast b\;\; (notazione\; infissa)\]
che verifica i punti 1, 3, 4 (vedere proposizioni operazioni su $\mathbb{Z}$) si chiama \(gruppo(S,\ast)\).
\\L'operazione su S è dunque:
\begin{enumerate}
    \item associativa
    \item con elemento neutro \textit{e}: \(\forall x, x\ast e=e\ast x=x\)
    \item per ogni elemento \textit{x} esiste un inverso rispetto al prodotto \(\ast\) cioè un elemento \textit{y} tale che \(x\ast y=y\ast x=e\), che si denota \(x^{-1}\)
\end{enumerate}

\subsection{Gruppo commutativo (abeliano)} 
Se il gruppo \((S, \ast)\) soddisfa anche la proprietà 2 (quindi associatività,  elemento neutro, opposto, +commutatività).

\subsection{Anello} 
Un anello è una terna \((A,+,\cdot)\) con:
\begin{enumerate}
    \item A insieme non vuoto
    \item \(+ \; \cdot\) due operazioni binarie, associative
    \item \((A,+)\) è un gruppo abeliano
    \item Distributività: \(\forall a, b, c \in A, \; a\cdot (b+c)=a\cdot b+a\cdot c\)
\end{enumerate}

\subsubsection{Anello commutativo} 
Se un anello \((A,+,\cdot)\) il prodotto è commutativo, cioè se \(\forall a,b\in A,\;a\cdot b=b\cdot a\).

\subsubsection{Anello unitario} 
Se esiste un elemento di A, che si denota con \(1_A\), tale che \(a\cdot 1_A=1_A\cdot a=a\).

\subsubsection{Divisore dello zero} 
Un elemento \(a\in A,\; a\neq0_A\) di un anello di dice divisore dello zero se esiste \(b\in A,b\neq 0\) con \(a\cdot b=0_A\).

\subsubsection{Dominio di integrità} 
Se \((A,+,\cdot)\) è privo di divisori dello zero.

\subsubsection{Legge di annullamento del prodotto} 
Se in un dominio di integrità \(a\cdot b=0_A\) allora \(a=0_A\) oppure \(b=0_A\).

\subsection{Campo}
Un campo è una terna \((K,+,\cdot)\) con \textit{K} insieme non vuoto e 2 operazioni.
\begin{itemize}
    \item \((K,+,\cdot)\) anello commutativo unitario
    \item Detto \(0_k\) l'elemento neutro della somma e denotato con \(K^*=K\setminus\{0_k\}\), deve valere che \(\forall x\in K^*:x\cdot x^{-1}=1_k\) \textit{(è un gruppo)}
\end{itemize}
Quindi campo \(\Leftrightarrow\) anello commutativo unitario con in più \(K\setminus\{0_k\}=(K^*,\cdot)\) gruppo.

\subsection{Semigruppo}
Sia X un insieme non vuoto, data $*$: 
\[X\;*\;X\rightarrow X\]
\[(a,b)\mapsto a * b\]
una operazione binaria associativa: \(\forall a,b,c\in X: a*(b*c)=(a*b)*c\)
\\
Un insieme \textit{X}, munito di una operazione associativa si chiama \textbf{semigruppo}.

\subsubsection{Monoide}
Se \((X,*)\) è un semigruppo ed inoltre esiste un elemento \(1_X\) tale che \(a*1_X=1_X*a=a\) (\(1_X\) elemento neutro dell'operazione \textit{*}), allora \((X,*)\) si chiama \textbf{monoide}.

\subsection{Elenco gruppi}
\textbf{\((A^*, \cdot )\)} è un monoide non commutativo.
\\\textbf{\((\mathbb{N}, +) \)} (commutativo) monoide (0 el. neutro) ma non è un gruppo.
\\\textbf{\( (\mathbb{Z}, +)\)} gruppo commutativo (0 el. neutro).
\\\textbf{\((\mathbb{Q}, +) \)} gruppo commutativo (0 el. neutro); \(\frac{p}{a}\rightarrow\;opposto\;-\frac{p}{q}\).
\\\textbf{\((\mathbb{N}^*, \cdot)\)} monoide, non è un gruppo.
\\\textbf{\((\mathbb{Z}^*, \cdot)\)} monoide, non è un gruppo.
\\\textbf{\((\mathbb{Q}, \cdot)\)} non è un gruppo, 0 non ha inverso.
\\\textbf{\((\mathbb{Q}^*,\cdot)\)} gruppo.
\\\textbf{\((\mathbb{R}, +)\)} gruppo.
\\\textbf{\((\mathbb{R}^*, \cdot)\)} monoide, gruppo.
\\\textbf{\((\mathbb{Z}_n, +)\)} gruppo finito commutativo; el. neutro \(\overline{0}\).
\\\textbf{\((\mathbb{Z}_n,\cdot)\)} monoide, semigruppo (non è un gruppo \(\overline{0}\) non è invertibile).
\\\textbf{\((\cup (\mathbb{Z}_n),\cdot) \)} gruppo, el. neutro \(\overline{1}=\{\overline{a}: (a,n)=1\}\) (el. invertibili).

\subsection{Gruppo simmetrico}

\subsubsection{Permutazione}
\(f:[n]\rightarrow [n]\) si chiama permutazione di \textit{n elementi} se \textit{f} è biiettiva.

\subsubsection{\(S_n\)}
\[S_n:=\{\sigma : [n]\rightarrow[n] : \sigma\;e'\;biiettiva\}\]
\[=\{\sigma : \sigma\;e'\;una\;biiezione\}\]

\subsubsection{Proposizione}
\[|S_n|=n!\]

\subsubsection{Proposizione}
\((S_n,\cdot)\) l'insieme delle permutazioni di \textit{n} elementi con il prodotto di composizione funzionale è un gruppo di cardinalità \textit{n!} non commutativo.
\\
\textbf{Dimostrazione}
\begin{itemize}
    \item \(S_n\) non vuoto, \(n\geq 1\)
    
    \item Esiste un elemento neutro rispetto al prodotto \(\cdot\), la permutazione identica: \(\sigma\circ id=id\circ\sigma=\sigma\).
    
    \item Prodotto associativo \(\forall\sigma ,\tau ,\rho\in S_n\) \((\sigma\circ \tau )\circ\rho (i)=\sigma\circ (\tau\circ\rho )(i)=\sigma ( \tau (\rho (i)))\)
    
    \item \(\forall\sigma\in S_n\) esiste un elemento \(\sigma ^{-1}\) tale che \(\sigma\circ\sigma ^{-1}=id\).
\end{itemize}

\subsubsection{\(3^a\) notazione: Permutazione come prodotto di cicli disgiunti}
\(S_n\): Definire una relazione di equivalenza su \([n]\) associata a \(\sigma \in S_n\).
\[x,y\in [n]\]
\[x\equiv _\sigma y\Leftrightarrow \exists i : y=\sigma ^i(x)\]
Si osservi che \(\sigma\in S_n\), allora la potenza \textit{i-esima} di \(\sigma\), con \(i\in\mathbb{N}\) è la permutazione \(\sigma ^i =\sigma\circ ...\circ\sigma\) per \textit{i} volte.

 \subsubsection{Orbita} L'orbita di \(x\in [n]\) è la classe di equivalenza di \textit{x} nella relazione \(\equiv _\sigma\). \[O_\sigma (x) =\{y\in [n]\;\exists i\;con\;y=\sigma ^i(x)\}\]


\subsubsection{Proposizione}
Se \(\tau _1\) e \(\tau _2\) hanno cicli disgiunti \(\tau _1\circ\tau _2 = \tau _2\circ\tau _1\)

\subsubsection{Permutazione ciclica}
Chiamo ciclica una permutazione di $S_n$ in cui nella rappresentazione in cicli disgiunti ha al più un solo ciclo di lunghezza\(>1\)

\subsubsection{Teorema prodotto di scambi}
Ogni permutazione si può scrivere come prodotto di scambi
\\\\
\textbf{Dimostrazione 1}: Se la permutazione ha un solo ciclo \(\sigma =(a_1, a_2, ... , a_k)=\) un k-ciclo = \((a_1,a_k)(a_1,a_{k-1})...(a_1,a_3)(a_1,a_2)=(a_1,a_2,a_3,...,a_k)\)
\\
\textbf{Dimostrazione 2}: Se ho un \(\sigma\) qualunque, allora
\[\sigma=C_1\cdot C_2\cdot ... \cdot C_k\]
dove \(C_i\) è un ciclo (nella decomposizione in cicli disgiunti)
\[C_1=(a_1,...,a_r)=(a_1,a_r)(a_1,a_{r-1})...(a_1,a_2)\]
\[C_2=(b_1,...,b_j)=(b_1,b_j)(b_1,b_{j-1})...(b_1,b_2)\]
\[. . .\]
\[\sigma =(a_1,a_r)(a_1,a_{r-1})...(a_1,a_2)\;(b_1,b_j)(b_1,b_{j-1})...(b_1,b_2) \]

\subsubsection{Teorema parità}
Il numero di scambi usati in diverse fattorizzazioni di una permutazione ha sempre la stessa parità.

\subsubsection{Pari, dispari}
Una permutazione è pari se il numero di scambi (in una sua fattorizzazione in scambi) è pari, dispari altrimenti.

\subsubsection{Gruppo alterno}

Le premutazioni pari si chiamano \textit{gruppo alterno}.

\subsubsection{Segno}

Data \(\sigma\) in \(S_n\), il segno di \(\sigma\) è \(\varepsilon (\sigma)=(-1)^{parita'\;di\;(\sigma)}\)

\subsection{Classi coniugate in \(S_n\)}

\subsubsection{Definizione}
Dato \(G\) gruppo rispetto ad un'operazione \(\cdot\), un elemento \(x'\) si dice coniugato con \(x\Leftrightarrow\)
\[\exists y\in G\;con\;:x'=yxy^{-1}\] 
In \(S_n\) \(\sigma ,\sigma '\) sono coniugate \(\Leftrightarrow\)
\[\exists\tau\in S_n: \sigma '=\tau\sigma\tau ^{-1}\]
(si dice che \(\sigma'\) è coniugato a \(\sigma\) tramite \(\tau\)) %controllare questa definizione
\\TODO: controllare correttezza definizione

\subsubsection{Proposizione}
Due permutazioni \(\sigma, \sigma '\in S_n\) sono coniugate \(\Leftrightarrow\) hanno la stessa struttura ciclica.

\subsection{Definizione multinsieme}
Una partizione \(\lambda\) di un intero \(n\) è un multinsieme di naturali \(\geq 1\) la cui somma da \(n\).

\subsection{Gruppi finiti}

\subsubsection{Proprietà 1}

Dato \((G,\cdot)\) gruppo e \(x,y\in G\) allora \((x\cdot y)^{-1}=y^{-1}\cdot x^{-1}\) (l'inverso del prodotto è il prodotto degli inversi in ordine inverso).
\\\\
\textbf{Dimostrazione:} \((xy)^{-1}=^? e_G\) (el. neutro del gruppo).
\\Ora 
\[(x\cdot y)^{-1}\cdot (y^{-1}\cdot x^{-1})=\]
\[x\cdot (y\cdot y^{-1})\cdot x^{-1}=\]
\[x\cdot e_G\cdot x^{-1}=\]
\[x\cdot x^{-1}=\]
\[e_G\]

\subsubsection{Proprietà 2}
In un gruppo vale sempre la cancellazione:
\[ax=bx\Leftrightarrow a=b\]
\\\\
\textbf{Dimostrazione:} \(\exists x^{-1}:\) Se \(ax=bx\) e moltiplico per \(x^{-1}\)
\[axx^{-1}=bxx^{-1}\]
\[a\cdot e=b\cdot e\]
\[a=b\]
\\\\\textit{Conseguenza:} Su una riga (qualunque) della tavola moltiplicativa del gruppo ci sono una e una sola volta tutti gli elementi del gruppo.

\subsection{Sottogruppi}

\subsubsection{Definizione}
Un sottogruppo \(S\) di \((G,\cdot)\) è:
\begin{itemize}
	\item Un sottoinsieme non vuoto di \(S\subseteq G\)
	\item \(S\), con la stessa operazione di \(G\) è un gruppo
\end{itemize}

\subsubsection{Criteri di verifica}
Per verificare che \(S\) sia un sottogruppo di \(G\);
\begin{itemize}
	\item Associatività: \textit{"gratis"} : \(S\subseteq G\) e il prodotto in \(G\) è associativo.
\end{itemize}

\begin{enumerate}

	\item \(\forall a,b \in S:a\cdot b\in S\) ovvero \(S\times S\rightarrow S\)

	\item \(e_G\in S\)

	\item \(\forall a\in S\subseteq G\),  \(a^{-1}\in S\)
\end{enumerate}

\subsubsection{Notazione}
\[(S,\cdot)\leq (G,\cdot )\]
altrimenti
\[S\leq G\]

\subsubsection{Proposizione}
\(S\) non vuoto e \(S\subseteq (G,\cdot)\) è un sottogruppo di \(G\) se e solo se
\[\forall\;a,b \in S: a\cdot b^{-1}\in S\;\;(*)\]
\textbf{Dimostrazione}
\\\textit{Ipotesi:} \(\forall a,b: a\cdot b^{-1}\in S\)
\\\textit{Tesi:} valgono 1, 2, 3 dei criteri di verifica.
\\\\
Dimostrazione 2:
\\
\(S\neq\emptyset :\exists a_0\in S\) applico \((*)\) ad \(a_0, a_0\):
\[a_0\cdot a_0^{-1}=e_G\;\in S\]
è quindi l'elemento neutro.
\\\\
Dimostrazione 3:
\\
\(\forall a\in S:a^{-1}\in S\)?
Per 2. \(e_G\in S, a\in S\), applico \((*)\)
\[e_G\cdot a^{-1}=a^{-1}\;\in S\]
\\
Dimostrazione 1:
\\
Dati \(a, b\in S\), \(a\cdot b\in S\)? Per la 3 \(b^{-1}\in S\).
\\
Dati \(a,b^{-1}\) per \((*)\)
\[a\cdot (b^{-1})^{-1}=a\cdot b\;\in S\]

\subsection{Proposizione: intersezione di sottogruppi}
Sia \((G,\cdot)\) un gruppo e \(H\leq G, K\leq G\) due sottogruppi. Allora:
\[H\cap K\leq G\]
L'intersezione di sottogruppi di \(G\) è un sottogruppo di \(G\)
\\
\\\textbf{Dimostrazione:}
\\
\begin{enumerate}
	\item \(1_G\in H\cap K\)?
	\\Poiche \(H\) e \(K\) sono sottogruppi \(1_G\in H,K\) e quindi \(1_G\in H\cap K\)

	\item Siano \(x, y\in H\cap K\): verifico che \(x\cdot y\in H\cap K\).
	\\\(x\in H\;e\;x\in K\); \(y\in H\;e\;y\in K\) allora:
	\[xy\in H;\;xy\in K\Rightarrow xy\in H\cap K\]

	\item Se \(x\in H\cap K\Rightarrow x^{-1}\in H\cap K\)?
	\\\textit{La dimostrazione è simila a quella del punto precendente}

\end{enumerate}

\subsection{Proposizione 1}
\[H_1, H_2,...H_t\leq G\Rightarrow H_1\cap H_2\cap ...\cap H_t\leq G\]

\subsection{Proposizione 2}
Siano \(S, T\leq G\):
\[S\cup T\leq G \Leftrightarrow S\cup T = T \lor S\cup T= S\]

\section{Sottogruppo generato}

\subsection{Definizione}
Siano \(G\) un gruppo e \(X\subseteq G\), si definisce sotto gruppo generato di \(X\) il più piccolo sottogruppo di \(G\) che contenga \(X\)

\subsection{Notazione}
\[\langle X \rangle\ := \bigcap _{X\subseteq H\leq G} H\]

\subsection{Proposizione}
Se \(X=\{x_,x_2...\}\subseteq G \neq 0\) allora:
\[\langle X\rangle=\{t_1,t_2,...,t_r:t_i\in X\;oppure\; t_i^{-1}\in X\}\]
L'insieme che contiene i prodotti finiti di elementi di \(X\) oppure i cui inversi sono in \(X\).
\\
\\\textbf{Dimostrazione:}
\begin{enumerate}
	\item \(\langle X \rangle\) contiene \(X\), \(r=1, t_i\in X\)

	\item \(\langle X\rangle\leq G\) 

	\begin{itemize}

		\item contiene \(1_G\): sia \(\overline{x}\in X\) qualunque \(\Rightarrow\overline{x}\in\langle X\rangle ,\overline{x}^{-1}\in\langle X\rangle\) e \(\overline{x}\cdot\overline{x}^{-1}=1_G\in\langle X\rangle\)

		\item \(\langle X\rangle\) è chiuso rispetto al prodotto di \(G\)

		\item Se \(t_1,t_2,...,t_r\in\langle X\rangle\), e \(t_1\)
		%ricontrollare questa parte di appunti

	\end{itemize}


\end{enumerate}

TODO:CONTROLLARE APPUNTI

\subsection{\(\langle X\rangle\) è il più piccolo sottogruppo che contiene \(X\)}
Da dimostrare in proprio, lo ha dato come esercizio

\subsection{Defizione: ordine (periodo)}
Se un elemento di \(G\) ha periodo finito, allora si chiama \textit{ordine} (o periodo) di \(g\) il più piccolo positivo tale che \(g^m=1_G\)

\subsection{Definizione: gruppo ciclico}
Un gruppo \(G\) si dice ciclico se esiste \(g_0\in G\) tale che \(G=\langle g_0\rangle\) \textit{(gruppo che viene generato da un solo elemento)}.

\subsection{Proposizione}
Il sottogruppo generato da un elemento (in un gruppo ciclico) è commutativo.
\\
\textbf{Dimostrazione:}
\[\langle g\rangle =\{g^h:h\in\mathbb{Z}\}\]
\[x=g^h,\;y=g^k\;\;\;h,k\in\mathbb{Z}\]
\[x\cdot y=g^hg^k=g^{h+k}=g^kg^h=y\cdot x\]

\subsection{Proposizione}
Sia \(G\) gruppo:
\begin{enumerate}

	\item Se \(g\in G\) ha periodo infinito \((\nexists \;h>0:g^h=e)\) allora \(\forall\; h,k\in\mathbb{Z}, h\neq k,\;g^h\neq g^k\): il gruppo ciclico generato da \(G\), \(\langle g\rangle \cong\mathbb{Z}\) (è isomorfo a $\mathbb{Z}$).

	\item \(g\) ha periodo finito.
	\\Se \(n\)=periodo di \(g=o(g)=ord_G(g)\) %non ho la o piccola in corsivo!%
	ovvero \(n=min\{k>0: g^k=e\}\) allora \(\langle g\rangle=\{e,g,g^2,...,g^{n-1}\}\) dove queste potenze sono tutte distinte.

\end{enumerate}
\textbf{Dimostrazione pt.1:} Dimostro che se:
\[g^h=g^k\Rightarrow h=k\]
infatti moltiplico per \(g^{-k}\) ed ho:
\[g^{h-k}=g^{k-k}\Rightarrow g^{h-k}=g^0=e\]
ma \(g\) è aperiodico 
\[\Rightarrow h-k=0\Rightarrow h=k\]
\textbf{Dimostrazione pt.2:} so che \(\langle g\rangle =\{g^h:h\in\mathbb{Z}\}\) devo dimostrare che ogni elemento \(g^h\) sta già in \(\{e,g,g^2,...g^{n-1}\}\).

Divido \(h\) per \(n\):
\[h=nq+r,\;\;0\leq r<n\]
\[\Rightarrow g^h=g^{nq+r}=g^{nq}g^r=(g^n)^qg^r=e^qg^r=eg^r=g^r\]
ed \(r\) è un numero \(0\leq r<n\) e quindi è una potenza dell'insieme.

\subsection{Proposizione: sottogruppi di un gruppo ciclico}
\begin{enumerate}
	
	\item[0.] Sottogruppi di \((\mathbb{Z},+)\): sono tutti e soli della forma \[H=m\mathbb{Z}=\{mh:h\in\mathbb{Z}=\langle m\rangle\},\;\;m\in\mathbb{N}\].
	\\\textit{Non dimostrato.}

	\item I sottogruppi di \(\langle g\rangle\) con \(g\in (G,\cdot)\), \(g\) aperiodico, sono tutti e soli della forma:
	\[H=\langle g^m\rangle\]
	per qualche \(m\in\mathbb{Z}\)
	\\\textit{Non dimostrato.}

	\item I sottogruppi di un gruppo ciclico generato da un elemento di ordine \(n\) (\(g^n=e\), \(n\) più piccolo positivo con \(g^n=e\)) sono anch'essi ciclici e generati da \(\langle g^h\rangle,\; h|n\).

\end{enumerate}

\subsection{Osservazione}
I sottogruppi di un gruppo ciclico finito verificano la seguente condizione:
\[H\leq\langle g\rangle\Rightarrow\vert H|\:| o(g)=|\langle g\rangle|\]
L'ordine di un sottogruppo \(H\leq\langle g\rangle\) divide l'ordine dell'elemento \(g\), che è anche l'ordine del gruppo.

\subsection{Proposizione}
In \(S_n\), sia \(\sigma (C_1)(C_2)...(C_k)\) la fattorizzazione di \(\sigma\) come prodotto dei suoi cicli disgiunti. Allora se \(m_i=\)lunghezza di \(C_i\)
\[ordine(\sigma)=mcm(m_1,m_2,...,m_k)\]

\subsection{Proposizione - Generatori gruppo cicliclo}
\(G=C_n=\langle g\rangle\) gruppo ciclico generato da un elemento di ordine \(n=\{id,g,g^2,...,g^n\}\). 
\\Tutti e soli i generatori di \(C_n\) sono le potenze di \(g\) con esponente coprimo con \(n\).
\\Generatori: \(g^t,\;(t,n)=1\)

\subsection{Teorema di Lagrange}
Se \(G\) è un gruppo finito, allora l'ordine di un sottogruppo divide l'ordine del gruppo:
\[H\leq G\Rightarrow |H|\,|\, |G| = o(H)|o(G)\] 
\\\textit{Oss: non vale sempre il viceversa.}

Se \(d|o(G)\Rightarrow\exists H\leq G,\; o(H)=d\)
\\\\\textbf{Dimostrazione:} Siano \(n=o(G)\) e \(m=o(H)\), \(i\) il numero di classi laterali destre modulo \(H\).
\\\(Ci_d=\) indice del sottogruppo \(H\) nel gruppo \(G\).
\\\(i=|G_{/_\sim d}|=\) numero di classi laterali.
Esistono \(a_1,a_2,...,a_i\) rappresentanti distinti delle classi laterali.
\[G=Ha_1\dot\cup Ha_2\dot\cup...\dot\cup Ha_i\Rightarrow |G|=o(G)=\]
\[=\sum ^i_{j=1} |Ha_j|=\sum ^i_{j=1} |H|=i\cdot |H|=i\cdot m\]
cioè ho \(n=i\cdot m\). \(ord(G)=\)numero classi laterali destre\(\cdot ord(H)\).
\\Da questa relazione deduco che:
\begin{enumerate}

	\item \(ord(H)|ord(G)\)

	\item \(i|o(G)\)

\end{enumerate}

\textit{Oss:} ripeto tutto per le classi laterali sinistre \(i_s\cdot m=n\).

\subsubsection{Corollario 1}
Se \(|G|=p\) primo, allora gli unici sottogruppi di \(G\) sono \(H=\{e\}\) oppure \(H=G\) (non ci sono sottogruppi intermedi).

\subsubsection{Corollario 2}
Se \(|G|=primo\), allora \(G\) è ciclico (in particolare è abeliano).
\\
\textbf{Dimostrazione:} Se \(|G|=p\) primo\(>1\).
\\Sia \(x_0\in G, x_0\neq e\). Sia \(H=\langle x_0\rangle\neq \{e\}\) (\(H=\{e,x_0,x_0^2...\})\), per il \textit{corollario 1}: 
\[H=G\Rightarrow G=\langle x_0\rangle\]

\subsection{Definizione: indice di un sottogruppo}
L'indice di un sottogruppo \(H\) in un gruppo \(G\) è:
\[i=i_s=i_d\]
e si denota:
\[i=[G:H]\]

\section{Classi laterali di un sottogruppo}
\subsection{Definizione: congruenza destra modulo}
Sia \((G,\cdot)\) un gruppo, sia \(H\leq G\) sottogruppo.

Definiamo congruenza destra modulo \(H\) la relazione così definita:
\[\forall\; a,b\in G: a\sim _d b\Leftrightarrow a\cdot b^{-1}\in H\]

\subsection{Proposizione}
\(\sim _d(mod\;H)\) è una relazione di equivalenza.
\\\textbf{Dimostrazione:}
\begin{itemize}

	\item (R) \(a\sim _d a\)?
	\[a\cdot a^{-1}=e\;\in H\]

	\item (S) \(a\sim _d b\Rightarrow b\sim _d a\;?\)
	\[ab^{-1}\in H\]
	\(H\) sottogruppo:
	\[(ab^{-1})^{-1}\;\in H\]
	\[\Rightarrow (b^{-1})^{-1}\cdot a^{-1}=b\cdot a^{-1}\Rightarrow b\sim _d a\]

	\item (T) \(a\sim _d b\) e \(b\sim _d c\Rightarrow a\sim _d c\;?\)
	\[ab^{-1}\in H\;e\;bc^{-1}\in H\] 
	\[(ab^{-1})(bc^{-1})\in H\]
	\(H\) è chiuso rispetto al prodotto
	\[(ab^{-1})(bc^{-1})=ac^{-1}\Rightarrow a\sim _d c\]

\end{itemize}

\subsection{Insieme quoziente}
Dato \(a\in G\): \([a]_{\sim _d} = H\cdot a\) dove \(Ha=\{ha:h\in H\}\), \(H=\{e,h_1,h_2...\}\), \(Ha=\{e\cdot a,h_1\cdot a,...\}\).
\\
\\\textbf{Dimostrazione:} devo provare 1.\(Ha\subseteq [a]_{\sim _d}\) e 2.\([a]_{\sim _d}\subseteq Ha\).
\begin{enumerate}

	\item \[b\in Ha\]
	\[\Leftrightarrow \exists h:b=ha\]
	\center moltiplicando per \(a^{-1}\)
	\[\Leftrightarrow h=ba^{-1}\]
	\[\Leftrightarrow ba^{-1}\in G\]
	\[\Leftrightarrow b\sim _d a\Leftrightarrow b\in [a]_{\sim _d}\]

	\item è la stessa di sopra ma partendo dalla fine verso l'inizio.

\end{enumerate}

\subsection{Proposizione}
Tutte le classi laterali destre hanno la stessa cardinalità.
\\
\\\textbf{Dimostrazione:} dimostro che \(|Ha|=|H|\;\forall a\in A\) \((|Ha|=[a]_{\sim _d}\), per transitività \(|Ha|=|Hb|\).
\\Sia \[\varphi :H\rightarrow Ha\]
\[h\rightarrow ha\]

\begin{itemize}

	\item Suriettiva: ogni elemento di \(Ha\) è del tipo \(ha\) per qualche \(h\in H\).

	\item Iniettiva: \(\varphi (a)=\varphi (h')\Rightarrow ha=h'a\Rightarrow\) per la cancellatività nel gruppo \(\Rightarrow h=h'\)

\end{itemize}

\subsection{Definizione: congruenza sinistra modulo}
\[\forall \;a,b\in G,\;\;a\sim _s b\Leftrightarrow b^{-1}a\in H\]
La classe laterale sinistra : \([a]_{\sim _s}=aH=\{ah: h\in H\}\)
