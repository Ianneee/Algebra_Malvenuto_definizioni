\section{Strutture algebriche}

\subsection{Gruppo} 
Un insieme S non vuoto, munito di una operazione \[m:S\times S\rightarrow S\]
\[(a,b)\mapsto m(a,b)=a\ast b\;\; (notazione\; infissa)\]
che verifica i punti 1, 3, 4 si chiama \(gruppo(S,\ast)\).
\\L'operazione su S è dunque:
\begin{itemize}
    \item associativa
    \item con elemento neutro \textit{e}: \(\forall x, x\ast e=e\ast x=x\)
    \item per ogni elemento \textit{x} esiste un inverso rispetto al prodotto \(\ast\) cioè un elemento \textit{y} tale che \(x\ast y=y\ast x=e\), che si denota \(x^{-1}\)
\end{itemize}

\subsection{Gruppo commutativo (abeliano)} 
Se il gruppo \((S, \ast)\) soddisfa anche la proprietà 2 (quindi associatività,  elemento neutro, opposto, +commutatività).

\subsection{Anello} 
Un anello è una terna \((A,+,\cdot)\) con:
\begin{itemize}
    \item A insieme non vuoto
    \item \(+ \; \cdot\) due operazioni binarie, associative
    \item \((A,+)\) è un gruppo abeliano
    \item Distributività: \(\forall a, b, c \in A, \; a\cdot (b+c)=a\cdot b+a\cdot c\)
\end{itemize}

\subsubsection{Anello commutativo} 
Se un anello \((A,+,\cdot)\) il prodotto è commutativo, cioè se \(\forall a,b\in A,\;a\cdot b=b\cdot a\).

\subsubsection{Anello unitario} 
Se esiste un elemento di A, che si denota con \(1_A\), tale che \(a\cdot 1_A=1_A\cdot a=a\).

\subsubsection{Divisore dello zero} 
Un elemento \(a\in A,\; a\neq0_A\) di un anello di dice divisore dello zero se esiste \(b\in A,b\neq 0\) con \(a\cdot b=0_A\).

\subsubsection{Dominio di integrità} 
Se \((A,+,\cdot)\) è privo di divisori dello zero.

\subsubsection{Legge di annullamento del prodotto} 
Se in un dominio di integrità \(a\cdot b=0_A\) allora \(a=0_A\) oppure \(b=0_A\).

\subsection{Campo}
Un campo è una terna \((K,+,\cdot)\) con \textit{K} insieme non vuoto e 2 operazioni.
\begin{itemize}
    \item \((K,+,\cdot)\) anello commutativo unitario
    \item Detto \(0_k\) l'elemento neutro della somma e denotato con \(K^*=K\setminus\{0_k\}\), deve valere che \(\forall x\in K^*:x\cdot x^{-1}=1_k\)
\end{itemize}
Quindi campo \(\Leftrightarrow\) anello commutativo unitario con in più \(K\setminus\{0_k\}=(K^*,\cdot)\) gruppo.

\subsection{Semigruppo}
Sia X un insieme non vuoto. 
\\\textit{*}:
\[X\;*\;X\rightarrow Z\]
\[(a.b)\mapsto a * b\]
una operazione binaria associativa: \(\forall a,b,c\in X: a+(b+c)=(a+b)+c\)
\\
Un insieme \textit{X}, munito di una operazione associativa si chiama \textbf{semigruppo}.

\subsubsection{Monoide}
Se \((X,+)\) è un semigruppo ed inoltre esiste un elemento \(1_X\) tale che \(a+1_X=1_X*a=a\) (\(1_X\) elemento neutro dell'operazione \textit{*}), allora \((X,+)\) si chiama \textbf{monoide}.

\subsection{Elenco gruppi}
\textbf{\((A^*, \cdot )\)} è un monoide non commutativo.
\\\textbf{\((\mathbb{N}, +) \)} (commutativo) monoide (0 el. neutro) ma non è un gruppo.
\\\textbf{\( (\mathbb{Z}, +)\)} gruppo commutativo (0 el. neutro).
\\\textbf{\((\mathbb{Q}, +) \)} gruppo commutativo (0 el. neutro); \(\frac{p}{a}\rightarrow\;opposto\;-\frac{p}{q}\).
\\\textbf{\((\mathbb{N}^*, \cdot)\)} monoide, non è un gruppo.
\\\textbf{\((\mathbb{Z}^*, \cdot)\)} monoide, non è un gruppo.
\\\textbf{\((\mathbb{Q}, \cdot)\)} non è un gruppo, 0 non ha inverso.
\\\textbf{\((\mathbb{Q}^*,\cdot)\)} gruppo.
\\\textbf{\((\mathbb{R}, +)\)} gruppo.
\\\textbf{\((\mathbb{R}^*, \cdot)\)} monoide, gruppo.
\\\textbf{\((\mathbb{Z}_n, +)\)} gruppo finito commutativo; el. neutro \(\overline{0}\).
\\\textbf{\((\mathbb{Z}_n,\cdot)\)} monoide, semigruppo (non è un gruppo \(\overline{0}\) non è invertibile).
\\\textbf{\((\cup (\mathbb{Z}_n),\cdot) \)} gruppo, el. neutro \(\overline{1}=\{\overline{a}: (a,n)=1\}\) (el. invertibili).

\subsection{Gruppo simmetrico}

\subsubsection{Permutazione}
\(f:[n]\rightarrow [n]\) si chiama permutazione di \textit{n elementi} se \textit{f} è biiettiva.

\subsubsection{\(S_n\)}
\[S_n:=\{\sigma : [n]\rightarrow[n] : \sigma\;è\;biiettiva\}\]
\[=\{\sigma : \sigma\;è\;una\;biiezione\}\]

\subsubsection{Proposizione}
\[|S_n|=n!\]

\subsubsection{Proposizione}
\((S_n,\cdot)\) l'insieme delle permutazioni di \textit{n} elementi con il prodotto di composizione funzionale è un gruppo di cardinalità \textit{n!} non commutativo.
\\
\textbf{Dimostrazione}
\begin{itemize}
    \item \(S_n\) non vuoto, \(n\geq 1\)
    
    \item Esiste un elemento neutro rispetto al prodotto \(\cdot\), la permutazione identica: \(\sigma\circ id=id\circ\sigma=\sigma\).
    
    \item Prodotto associativo \(\forall\sigma ,\tau ,\rho\in S_n\) \((\sigma\circ \tau )\circ\rho (i)=\sigma\circ (\tau\circ\rho )(i)=\sigma ( \tau (\rho (i)))\)
    
    \item \(\forall\sigma\in S_n\) esiste un elemento \(\sigma ^{-1}\) tale che \(\sigma\circ\sigma ^{-1}=id\).
\end{itemize}

\subsubsection{\(3^a\) notazione: Permutazione come prodotto di cicli disgiunti}
\(S_n\): Definire una relazione di equivalenza su \([n]\) associata a \(\sigma \in S_n\).
\[x,y\in [n]\]
\[x\equiv _\sigma y\Leftrightarrow \exists i : y=\sigma ^i(x)\]
Si osservi che \(\sigma\in S_n\), allora la potenza \textit{i-esima} di \(\sigma\), con \(i\in\mathbb{N}\) è la permutazione \(\sigma ^i =\sigma\circ ...\circ\sigma\) per \textit{i} volte.

 \subsubsection{Orbita} L'orbita di \(x\in [n]\) è la classe di equivalenza di \textit{x} nella relazione \(\equiv _\sigma\). \[O_\sigma (x) =\{y\in [n]\;\exists i\;con\;y=\sigma ^i(x)\}\]


\subsubsection{Proposizione}
Se \(\tau _1\) e \(\tau _2\) hanno cicli disgiunti \(\tau _1\circ\tau _2 = \tau _2\circ\tau _1\)

\subsubsection{Permutazione ciclica}
Chiamo ciclica una permutazione di \(S_n)\) in cui nella rappresentazione in cicli disgiunti ha al più un solo ciclo di lunghezza\(>1\)

\subsubsection{Teorema prodotto di scambi}
Ogni permutazione si può scrivere come prodotto di scambi
\\\\
\textbf{Dimostrazione 1}: Se la permutazione ha un solo ciclo \(\sigma =(a_1, a_2, ... , a_k)=\) un k-ciclo = \((a_1,a_k)(a_1,a_{k-1})...(a_1,a_3)(a_1,a_2)=(a_1,a_2,a_3,...,a_k)\)
\\
\textbf{Dimostrazione 2}: Se ho un \(\sigma\) qualunque, allora
\[\sigma=C_1\cdot C_2\cdot ... \cdot C_k\]
dove \(C_i\) è un ciclo (nella decomposizione in cicli disgiunti)
\[C_1=(a_1,...,a_r)=(a_1,a_r)(a_1,a_{r-1})...(a_1,a_2)\]
\[C_2=(b_1,...,b_j)=(b_1,b_j)(b_1,b_{j-1})...(b_1,b_2)\]
\[. . .\]
\[\sigma =(a_1,a_r)(a_1,a_{r-1})...(a_1,a_2)\;(b_1,b_j)(b_1,b_{j-1})...(b_1,b_2) \]

\subsubsection{Teorema parità}
Il numero di scambi usati in diverse fattorizzazioni di una permutazione ha sempre la stessa parità.

\subsubsection{Pari, dispari}
Una permutazione è pari se il numero di scambi (in una sua fattorizzazione in scambi) è pari, dispari altrimenti.

\subsubsection{Gruppo alterno}

Le premutazioni pari si chiamano \textit{gruppo alterno}.

\subsubsection{Segno}

Data \(\sigma\) in \(S_n\), il segno di \(\sigma\) è \(\varepsilon (\sigma)=(-1)^{parita'\;di\;(\sigma)}\)

\subsection{Gruppi finiti}

\subsubsection{Proprietà 1}

Dato \((G,\cdot)\) gruppo e \(x,y\in G\) allora \((x\cdot y)^{-1}=y^{-1}\cdot x^{-1}\) (l'inverso del prodotto è il prodotto degli inversi in ordine inverso).
\\\\
\textbf{Dimostrazione:} \((xy)^{-1}=^? e_G\) (el. neutro del gruppo).
\\Ora 
\[(x\cdot y)^{-1}\cdot (y^{-1}\cdot x^{-1})=\]
\[x\cdot (y\cdot y^{-1})\cdot x^{-1}=\]
\[x\cdot e_G\cdot x^{-1}=\]
\[x\cdot x^{-1}=\]
\[e_G\]

\subsubsection{Proprietà 2}
In un gruppo vale sempre la cancellazione:
\[ax=bx\Leftrightarrow a=b\]
\\\\
\textbf{Dimostrazione:} \(\exists x^{-1}:\) Se \(ax=bx\) e moltiplico per \(x^{-1}\)
\[axx^{-1}=bxx^{-1}\]
\[a\cdot e=b\cdot e\]
\[a=b\]
\\\\\textit{Conseguenza:} Su una riga (qualunque) della tavola moltiplicativa del gruppo ci sono una e una sola volta tutti gli elementi del gruppo.

\subsection{Sottogruppi}