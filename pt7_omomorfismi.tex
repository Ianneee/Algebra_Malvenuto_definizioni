\section{Omomorfismi}

\subsection{Isomorfismo}
Dati \((G,*)\) e \((H,\cdot)\) due gruppi, un isomorfismo di \(G\) in \(H\) è
\begin{itemize}
	\item \(\varphi :G\rightarrow H\) una biiezione.
	\item \(\varphi\) rispetta le operazioni di gruppo, cioè:
	\[\forall\; a,b \in G: \varphi (a*b)=\varphi(a)\cdot\varphi (b),\;\; \varphi(a)\;e\;\varphi (b)\in H\]
\end{itemize}
Si dice che \(G\) è isomorfo ad \(H\) e si scrive \(G\cong H\).

\subsection{Omomorfismo}
Se \(\varphi :G\rightarrow H\) conserva le operazioni di \(G\) e \(H\), \(\varphi\) si chiama omomorfismo.

\subsection{Epimorfismo}
Se \(\varphi\) è suriettiva, \(\varphi\) si chiama epimorfismo.

\subsection{Monomorfismo}
Se \(\varphi\) è iniettiva, si chiama monomorfismo.

\subsection{Isomorfismo 2}
Se \(\varphi\) è biunivoca, allora \(\varphi\) si chiama isomorfismo.

\subsection{Proposizione}
L'isomorfismo tra gruppi è una relazione di equivalenza.