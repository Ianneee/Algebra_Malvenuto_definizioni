\section{Capitolo 1}
Relazione e corrispondenza sono interscambiabili.

\subsection{Corrispondenza}
Una corrispondenza \(\rho\) di X in Y è una terna ( \(\rho, X, Y\)) dove \(\rho \subseteq X\times Y\).

\subsection{Relazione in se}
Una Relazione di \(X\) in sè, è una corrispondenza \(\rho\) di X in X.
Se \((x, y) \in\rho\) si scrive anche \(x\rho y\)(notazione infissa), cioè x è in relazione \(\rho\) con y.

\subsection{Relazione/Corrispondenza inversa} 
Una corrispondenza \(\rho\) di X in Y è la relazione di Y in X denotata con \(\rho ^{-1}\) data dalla seguente:
\[y\rho^{-1}x \Leftrightarrow x\rho y\]

\subsection{Relazione di equivalenza} una relazione su A (cioè un sottoinsieme \(\rho\) di AxA) si dice di equivalenza se verifica le tre seguenti proprietà:

\textit{Riflessiva:} \(\forall a \in A, a\rho a\).

\textit{Simmetrica:} \(\forall a,b\) in \(A, a\rho b \Rightarrow b\rho a\)

\textit{Transitiva:} \(\forall a, b, c \in A\) se \((a\rho b \wedge b\rho c) \Rightarrow a\rho c\)

\subsection{Relazione banale (di uguaglianza)} 
Su A \(x, y \in A\) \(x\rho y \Leftrightarrow x=y\)

\subsection{Relazione caotica} 
Su A \(x\rho y\; \forall x,y \in A\)

\subsection{Classe di equivalenza} 
Data la relazione \(\rho\) in A, si definisce classe di equivalenza modulo \(\rho\) di un elemento \(a \in A\) l'insieme di tutti gli elementi che sono equivalenti ad \(a\); si denota con \([a]_\rho\).

\[[x]_\rho :=\{y\in A : y\rho x\} \]

\subsection{Insieme quoziente} 
Data la relazione di equivalenza \(\rho\) su A, si definisce insieme quoziente l'insieme delle classi di equivalenza di \(\rho\) dato \(x\in A\) si denota con \(A/_\rho\).
\[A/_\rho = \{[x]_\rho : x \in A \} \]
\newline
Nota: Relazione di equivalenza e partizioni insiemistiche sono sostanzialmente la stessa cosa.

\subsection{Partizione insiemistica} 
Una partizione insiemeistica di A è una famiglia di sottoinsiemi di A non vuoti, tali che ad ogni elemento di A corrisponde un solo sottoinsieme.
\[H = \{A_i : i \in I \} \] con \[A_i \subseteq A\; \forall i \in I\]
con
\[i \neq j,\;\; i,j \in I \Leftrightarrow A_i \cap A_j=\emptyset \]
che equivale a dire:
\[\cup_{i\in I} \; A_i = A\]
cioè la famiglia H ricopre A.

\subsection{Funzione/Applicazione} 
\(f:S\rightarrow T\) è un'applicazione di S in T se (f, S, T) è una corrispondenza di S in T, ovvero \(f\subseteq S\times T\) che soddisfa la seguente proprietà:
\\
\(\forall x \in S \; \exists !\) y in T denotato con \(y=f(x)\), f è una legge univoca (ben definita).
\\\\
L'elemento f(x) si chiama \textbf{immagine dell'elemento}.
\\\\
\textbf{L'immagine di f} è un sottoinsieme del codominio T definito da:
\[Im(f) := \{y \in T : \exists\; x \in S, y=f(x)\}\]
\newline
\textbf{Controimmagine di y} è il sottoinsieme di S del dominio definito da:
\[f^{-1}(y):=\{x\in S:f(x)=y\}\subseteq S\]

\subsection{Iniettiva} 
f è iniettiva \(\Leftrightarrow \forall x, x' \in S :[f(x)=f(x')\Rightarrow x=x']\).
\\\textit{Definizione alternativa:} f è iniettiva \(\Leftrightarrow \forall x,x' \in S : [f(x)\neq f(x') \Rightarrow x\neq x']\).
\\
f è iniettiva \(\Leftrightarrow \forall y \in T \; |f^{-1}|\leq 1\), ovvero per ogni elemento y in T esiste al più un'immagine.

\subsection{Suriettiva} 
f è suriettiva se \(\Rightarrow \forall y\in T \;\;\exists\; x \in S : f(x)=y\)
\\
\textit{Definizione alternativa:} f è suriettiva \(\Leftrightarrow f(S) = Im(S) = T\).
\\
f è suriettiva \(\Leftrightarrow \forall y \in T \; |f^{-1}(y)|\geq 1\), ovvero per ogni elemento y in T esiste almeno un'immagine.

\subsection{Biunivoca (biiettiva)} se f è sia iniettiva che suriettiva.
\\
f è biiettiva \(\Leftrightarrow \forall y \in T \; |f^{-1}(y)|=1\), ovvero per ogni elemento y in T esiste una sola immagine.

\subsection{Funzione caratteristica} 
E' la funzione che vale 1 se \(x \in S\), 0 se \(x \notin S\).

\subsection{Operazione binaria} 
Un'operazione binaria su S, è un'applicazione \(m:S\times S \rightarrow S\); notazione funzionale \((s, s') \mapsto m(s, s')\); notazione infissa \(sms'\) o \(s*s\).

\subsection{Assiomi di Peano} per la costruzione dei naturali \(\mathbb{N}\)
\begin{enumerate}
    \item I numeri formano una classe
    \item Lo "zero" è un numero
    \item Se \(a\) è un numero allora il successore \(a'\) è un numero
    \item Se \(a\neq b\) sono due numeri allora \(a'\neq b'\)
    \item Lo "zero" non è successore di nessun numero (\(\nexists \; a\) numero tale che \(zero=a'\))
    \item Assioma di induzione:
    \\Se S è una classe di numeri tale che:
    \begin{itemize}
        \item \(zero\in S\)
        \item Se \(a\in S\) allora \(a'\in S\)
    \end{itemize}
    allora ogni naturale è in S.
\end{enumerate}
I naturali sono la più piccola classe che 
\begin{itemize}
    \item Contiene lo zero
    \item Chiusa rispetto a contenere i successori
\end{itemize}

\subsection{Principio del buon ordinamento di \(\mathbb{N}\)} 
Se \(S\subseteq \mathbb{N}, S\neq\emptyset\), allora esiste un minimo in S, cioè esiste \(m\in S\) tale che se \(h\in\mathbb{N}, h<m\) allora \(h\notin S\).

\subsection{Teor: Divisione con resto su \(\mathbb{N}\)} 
Siano \(a,b\in\mathbb{N}, b\neq 0\); allora esistono \(q, r\in\mathbb{N}\) tali che
\begin{itemize}
    \item \(a=bq+r\)
    \item \(0\leq r<b\)
\end{itemize}
\(\forall a,b\in\mathbb{Z}, b\neq 0; \exists\) unici \(q, r\in\mathbb{Z}\) con \(a=bq+r \land 0\leq r<b\)
\\\textbf{Dimostrazione} per induzione:
\\\textit{Base:} $P(0),\; a=0$:
\\\\$\exists !q,r$ con $0=qb+r$ e $0\leq r <b$?
\\Prendo $q=0,\; r=0$ e si ha $0=0b+0,\;\;0\leq r=0<b$.
\\\\\textit{Ipotesi induttiva:} $\forall\; 0\leq k<a$ esistono quoziente e resto.
\\\\\textit{Passo induttivo:} due casi:
\begin{enumerate}
  
\item $a<b$:
  \[\Rightarrow a=b0+a,\; 0\leq a=r<b\]

\item $a\geq b\;\;(b>0)$
  \\Consideriamo $a-b=a'$:
  \[0\leq a-b<a\]
  \[0\leq a' <a\]
  $a'$ è uno dei $k$: $\forall k: \; 0\leq k<a$ vale $P(k)\Rightarrow P(a')$ cioè esistono $q', r'$ tali che:
  \[a'=bq'+r',\;\;0\leq r'<b\]
  \[\Rightarrow a=a'+b\Rightarrow a=q'b+r'+b=\]
  \[=(q'+1)b+r',\;\;0\leq r'<b\]
\end{enumerate}

\section{Calcolo combinatorio}

\subsection{Notazione funzionale} 
Insieme delle applicazioni da A verso B
\[B^A=\{f:A\rightarrow B\}\]

\subsection{Fattoriale crescente} 
\[n^{(m)}:= n\ast (n+1)\ast ... \ast (n+m-1)\]

\subsection{Fattoriale decrescente} 
\[n_{(m)}:= n\ast (n-1)\ast ... \ast (n-m+1)\]

\subsection{Pigenhole principle (principio dei cassetti)} 
Se ho \textit{n} oggetti e \textit{m} cassetti, se \(n>m\) e devo disporre tutti gli oggetti nei cassetti allora esiste un cassetto che contiene almeno due oggetti.

\subsection{Permutazione} 
Sia A un insieme. Una biiezione \(f:A\rightarrow A\) si chiama anche \textit{permutazione} di A.

\subsection{Coefficiente binomiale}
\textbf{Prima interpretazione combinatoria:} \(\binom{n}{i}\) è il coefficiente di \(x^i y^{n-i}\) nello sviluppo \((x+y)^n=\sum _{z_i\in \{x,y\}} z_1 ...z_n\), ovvero il numero di stringhe binarie (su {x, y})
\begin{itemize}
    \item lunghe n
    \item con i occorrenze di x
    \item con n-i occorrenze di y
    \item \((x+y)^n =\sum _{i=0} ^n\binom{n}{i} x^iy^{n-i}\)
\end{itemize}
\noindent\textbf{Seconda interpretazione combinatoria:} numero di sottoinsiemi di cardinalità \textit{i} su un insieme \([n]\) di cardinalità \textit{n}.

\subsection{Formula} 
\[\binom{n}{i}=\frac{n(n-1)\ast ... \ast (n-i+1)}{i!} = \frac{n!}{i!(n-i)!}\]

\subsection{Relazione ricorsiva}
\[\binom{n}{i}=\binom{n-1}{i-1}+\binom{n-1}{i}\]
Dimostrazioni algebrica e combinatoria.

\subsection{Simmetria}
\[\binom{n}{i}=\binom{n}{n-i}\]
Il coefficiente binomiale è simmetrico rispetto al centro della riga n-esima \(\lfloor\frac{n}{2}\rfloor\) del triangolo rappresentante tutti i coefficienti del coefficiente binomiale.

Dimostrazioni algebrica e combinatoria.

\subsection{Relazione d'ordine}
Una relazione \(\rho\) su \textit{X} è una relazione d'ordine (o un ordine, o un ordinamento) se valgono per \(\rho\) le proprietà:
\begin{itemize}
    \item (R) \(\forall x, x\rho x\)
    \item (AS) \(\forall x,y\; (x\rho y\land y\rho x)\Rightarrow x=y\)
    \item (T) \(\forall x,y,z\; (x\rho y\land y\rho z)\Rightarrow x\rho z\)
\end{itemize}

\subsection{Proposizione - Naturali e divisibilità}
$(\mathbb{N}, |)$ l'insieme dei naturali con la divisibilità è un insieme parzialmente ordinato. (La divisibilità è una relazione d'ordine su $N*$).
\\TODO: Dimostrazione R, AS, T

\subsection{POSET (Partial order set)}
Un insieme munito di una relazione d'ordine si dice parzialmente ordinato.
