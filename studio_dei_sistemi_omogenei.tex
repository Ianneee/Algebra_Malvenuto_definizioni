\section{Studio dei sistemi omogenei}

\subsection{Proposizione}
Sia $AX=0$ un sistema lineare omogeneo (con $A\in M_{m\times n}(\mathbb{R}$). Allora 
\[W=\{X=\begin{bmatrix}x_1\\\vdots\\x_n\end{bmatrix}\in\mathbb{R}:AX=0\}\]
L'insieme delle soluzioni del S.L.O. è un sottospazio di $R^n$
\\
\\\textbf{Dimostrazione}
Se $X, X'$ sono in $W$ allora $\alpha X+\beta X'\in W$?
$A(\alpha X+\beta X')=\alpha AX+\beta AX'=\alpha \underline{0}+\beta \underline{0}=\underline{0}$
\subsection{Teorema: Struttura sulle soluzioni di un sistema}
Sia $AX=b$ un sistema lineare ($b$ colonna dei termini noti). 
Sia $v_0\in\mathbb{R}^n$ una soluzione del sistema $AX=b$.
$v=\begin{bmatrix}v_1\\...\\v_n\end{bmatrix}$ (una soluzione "particolare")
Allora ogni altra soluzione di $AX=b$ si scrive nella forma 
\[v=v_0+w\;\;(soluzione\;generale)\]
al variare di $w\in W=\{X:AX=\underline{0}\}$ cioè al variare di $w$ nelle soluzioni del sistema lineare omogeneo associato $AX=\underline{0}$.

\[(Av_0=b,\;Aw=\underline{0}\Leftrightarrow A(v_0+w)=b\Rightarrow Av_0+Aw=b+\underline{0}=b)\]

\subsection{Nota importante}
$A(X+Y)=AX+AY$: la moltiplicazione per una matrice è quindi lineare.